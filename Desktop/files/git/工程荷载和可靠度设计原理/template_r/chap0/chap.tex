\ifx\allfiles\undefined
\documentclass[12pt, a4paper, oneside, UTF8]{ctexbook}
\def\path{../config}
\input{../config/_config}
\begin{document}
\input{../config/cover}
\else
\fi

\chapter{风荷载}

\begin{theorem}
$$
w = w_m - w_b = \frac{1}{2} \rho v^2 = \frac{1}{2} \frac{\gamma}{g} v^2
$$

风速与风压的关系公式,其中 $w_m$为最大风压力(速度降为0),$w_b$为原先风压力,$\gamma$ 为空气单位体积的重力,$g$ 为重力加速度。
\end{theorem}

基本风压的相关因素:

\begin{enumerate}
    \item 地貌(影响摩擦)
    \item 标高(一般取10m)
    \item 最大风速的样本时间:样本时间显然越长越大
    \item 公称风速的时距:公称风速实际是一定时间间隔的平均风速,不可过长也不可过短。
    \item 基本风速重现期的长短:{\color{red}该时间范围内的最大风速定义为基本风速}
\end{enumerate}

\begin{definition}
    根据实测结果分析,平均风速沿高度变化的规律可用指数函数来描述,即

$$
\frac{\bar{v}}{v_s} = \left( \frac{z}{z_s} \right)^\alpha
$$

式中 $\bar{v}$、$z$——任一点的平均风速和高度;

$v_s$、$z_s$——标准高度处的平均风速和高度(一般取10m)。

$\alpha$——与地貌或地面粗糙度有关的指数。
\end{definition}

由于远离地面的高度的风速相同,$$v_{0s} \left( \frac{H_{Ts}}{z_s} \right)^{\alpha_s} = v_{0a} \left( \frac{H_{Ta}}{z_a} \right)^{\alpha_a}$$
于是有
\[
v_{0a} = v_{0s} \left( \frac{H_{Ts}}{z_s} \right)^{\alpha_s} \left( \frac{H_{Ta}}{z_a} \right)^{-\alpha_a} 
\]

可得任意地貌的基本风压 \(w_{0a}\) 与标准地貌的基本风压 \(w_0\) 的关系为:

\[
w_{0a} = w_{0s} \left( \frac{H_{Ts}}{z_s} \right)^{2\alpha_s} \left( \frac{H_{Ta}}{z_a} \right)^{-2\alpha_a} 
\]

不同地表粗糙度有不同的梯度风高度,地面粗糙度小,风速变化快,梯度风高度比地面粗糙度大的地区低;反之,地面粗糙度越大,梯度风高度将越高。

\begin{remark}
    任一水平风作用在任意截面的细长物体表面上,会在其表面产生风压,将物体表面上的风压沿表面积分,将得到三种力的成分,即顺风向力 \( P_D \)、横风向力 \( P_L \) 及扭力矩 \( P_M \)。
    
    顺风向的风效应分解为平均风 (即稳定风)和脉动风(也称阵风脉动)。

    横风向力较顺风向力小得多,对于对称结构,横风向力更是可以忽略。
    然而,对于一些细长的柔性结构,横风向力可能会产生很大的动力效应横风向。
\end{remark}

\begin{definition}
    风载体型系数(和风速的变化相关,而风速变化由风流动的截面相关,所以和房屋几何形态相关):
    $$
    w=\mu_s w_s
    $$
    $$
    \mu_s=1-\frac{v^2}{v_{0}^2}
    $$
    其中$v$为变化后的风速,$v_0$为原先风压力
\end{definition}

所以一个显然的结论是,迎风面风被阻挡,风速变小,表现为正压力。

背风面横截面变宽,风速变大,表现为负压力,或者说吸力。

\begin{definition}
    风压高度变化系数
    \[
\mu_z(z) = \left( \frac{H_{Ts}}{z_s} \right)^{2\alpha_s} \left( \frac{H_{Ta}}{z_a} \right)^{-2\alpha_a} \left( \frac{z}{z_s} \right)^{2\alpha_a}
\]
\end{definition}

A类为海面、沙漠等表面光滑的地貌;B类为田野、乡村、房屋稀疏地等;
C类为密集建筑群城市市区;D类为密集建筑群且房屋较高市区

\begin{theorem}
    风荷载标准值的计算公式:

    $$w_k = \beta_r \mu_s \mu_z w_0$$
    $\beta$是风振系数,若不考虑风压脉动对结构顺风向风振的影响,计算取1

    这也解释了一句话,顺风荷载是平均风和脉动风的线性叠加
\end{theorem}



横风向风振验算

(1) 当 $Re < 3 \times 10^5$ 且结构顶部风速 $v_H$ 大于共振风速 $v_{cr}$ 时,可发生亚临界的微风共振。此时可在构造上采取防振措施,或控制结构的临界风速 $v_{cr}$ 不小于 15m/s。

(2) 当 $Re \geq 3.5 \times 10^6$ 且结构顶部风速 $v_H$ 的 1.2 倍大于共振风速 $v_{cr}$ 时,可发生跨临界的强风共振。此时应考虑横风向风振的等效风荷载。

(3) 当 $3 \times 10^5 \leq Re < 3.5 \times 10^6$ 时,则发生超临界范围的风振,可不作处理。{\color{red}同时此时的斯特罗哈数离散性比较强}。

\begin{definition}
    斯特罗哈数
    \[
    S_t = \frac{f_sD}{v}
    \]
    $f_s$描述脱落气流的频率,当其和物体的固有频率相等时,发生共振现象。
\end{definition}

\begin{remark}
    根据PPT,只有亚临界状态和跨临界状态我们认为是共振的。此时斯特罗哈数分别等于0.2和0.3。
    
    斯特罗哈数与建筑截面形状以及雷诺数相关。横向共振时,脱落频率一定。
\end{remark}

\begin{example}
    在横风向共振所处区域内:(D)

    A.斯托哈数接近于常数0.2; 

    B.斯托哈数离散性很大;

    C.风漩涡脱落频率与风速成正比; 

    D.风漩涡脱落频率保持常数.
\end{example}

\begin{example}
    杭州某高层建筑所在场地为C类地貌,已知杭州基本风压 $w_0 = 0.45\,\mathrm{kN/m}^2$,$\frac{\gamma}{2g} = \frac{1}{1740}$,试分别计算该场地 $50\,\mathrm{m}$ 处的风压及对应的风速值。

    \begin{figure}[H]
        \centering
        \includegraphics[width=0.5\textwidth]{../figure/3.png}
        \caption{地貌参数}
    \end{figure}

    死记硬背下公式,当然可以现场推导下:
    \[
    w_{0c} = w_{0s} \left( \frac{H_{Ts}}{z_s} \right)^{2\alpha_s} \left( \frac{H_{Tc}}{z_c} \right)^{-2\alpha_c}=0.45 \left( \frac{350}{10} \right)^{2 \times 0.15} \left( \frac{450}{10} \right)^{-2 \times 0.22} = 0.245 \,\mathrm{kN/m}^2
    \]
    于是,求得未知地区的标准风压后,计算其50m高度的风压:
    \[
    w_c = w_{0c} (\frac{50}{10})^{2 \times 0.22}=0.497\,\mathrm{kN/m}^2
    \]
    计算风速:
    \[
    w_c = \frac{1}{2} \rho v^2 = \frac{1}{2} \frac{\gamma}{g} v^2
    \]
    \[
    v = \sqrt{2 w_c g / \gamma} = \sqrt{0.497  / (1/1740)} = 29.4\,\mathrm{m/s}
    \]
\end{example}

\begin{example}
    已知:某矩形高层建筑,结构高度 $H=40\,\mathrm{m}$,平面长度 $D=30\,\mathrm{m}$,宽度 $B=25\,\mathrm{m}$,建造于城市市郊,地面粗糙度 $\alpha_a=0.22$,标准地貌的地面粗糙指数 $\alpha_s=0.15$,基本风压 $w_0=0.5\,\mathrm{kN/m}^2$。假设不考虑脉动风影响,沿高度均匀分成四段进行近似计算。求:顺风向风产生的建筑底部弯矩?

    \textbf{解:}

    1. 计算每段高度:
    \[
    \Delta h = \frac{H}{4} = \frac{40}{4} = 10\,\mathrm{m}
    \]
    各段中心高度分别为 $z_1=5\,\mathrm{m}$,$z_2=15\,\mathrm{m}$,$z_3=25\,\mathrm{m}$,$z_4=35\,\mathrm{m}$。

    2. 计算各段风压(采用风压高度变化系数):
    \[
    w(z) = \mu_s \mu_z(z) w_0
    \]
    其中 $\mu_s = 1.3 ,\text{是体型系数},\mu_z(z)=\left( \frac{H_{Ts}}{z_s} \right)^{2\alpha_s} \left( \frac{H_{Ta}}{z_a} \right)^{-2\alpha_a} \left( \frac{z}{z_s} \right)^{2\alpha_a} = 0.54 \times (\frac{z}{z_s})^{2 \times 0.22}$
    \begin{align*}
    w_1 &= 1.3 \times 0.54 \times (5/10)^{0.44} \times 0.5 = 0.26\,\mathrm{kN/m}^2 \\
    w_2 &= 1.3 \times 0.54 \times (15/10)^{0.44} \times 0.5 = 0.42\,\mathrm{kN/m}^2 \\
    w_3 &= 1.3 \times 0.54 \times (25/10)^{0.44} \times 0.5 = 0.52\,\mathrm{kN/m}^2 \\
    w_4 &= 1.3 \times 0.54 \times (35/10)^{0.44} \times 0.5 = 0.61\,\mathrm{kN/m}^2 \\
    \end{align*}
    3. 计算各段风荷载(作用面积 $A = B \times \Delta h = 25 \times 10 = 250\,\mathrm{m}^2$):
    \[
    F_i = w_i \cdot B \cdot \Delta h
    \]

    \begin{align*}
    F_1 &= 0.26 \times 250 = 65\,\mathrm{kN} \\
    F_2 &= 0.42 \times 250 = 105\,\mathrm{kN} \\
    F_3 &= 0.52 \times 250 = 130\,\mathrm{kN} \\
    F_4 &= 0.61 \times 250 = 152.5\,\mathrm{kN} \\
    \end{align*}
    4. 计算各段力臂(至底部距离):
    \begin{align*}
    l_1 &= 35\,\mathrm{m} \\
    l_2 &= 25\,\mathrm{m} \\
    l_3 &= 15\,\mathrm{m} \\
    l_4 &= 5\,\mathrm{m} \\
    \end{align*}

    5. 计算底部弯矩:
    \[
    M = \sum_{i=1}^{4} F_i \cdot l_i = 91.8 \times 35 + 148.5 \times 25 + 176.3 \times 15 + 197.5 \times 5
    \]
    \[
    = 3213 + 3712.5 + 2644.5 + 987.5 = 10557\,\mathrm{kN \cdot m}
    \]

    \textbf{答:} 顺风向风产生的建筑底部弯矩约为 $10557\,\mathrm{kN \cdot m}$。

\end{example}

\begin{example}
    什么情况下要考虑结构横风向风振效应?如何进行横风向风振验算?什么情况下要考虑结构横风向风振效应?如何进行横风向风振验算?

    当横向风作用引起结构共振时,结构横风向风振效应不可忽略。通过雷诺数验算。
\end{example}







\ifx\allfiles\undefined
\end{document}
\fi