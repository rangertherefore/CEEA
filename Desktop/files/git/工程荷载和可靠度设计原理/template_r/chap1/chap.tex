\ifx\allfiles\undefined
\documentclass[12pt, a4paper, oneside, UTF8]{ctexbook}
\def\path{../config}
\input{../config/_config}
\begin{document}
% \input{../config/cover}
\else
\fi

\chapter{荷载类型}

\begin{definition}
    作用(action):是指能使结构产生效应(effect)的各种原因的总称。

    其中作用又分为直接作用和间接作用,{\color{red}直接作用是狭义上的荷载}。

    可以简记为和力有关的都是直接作用,和变形有关的都是间接作用。
\end{definition}

\begin{example}
    地震引起的惯性力属于间接作用。

    但是由爆炸、离心作用等产生的作用在物体上的惯性力以及制动力是荷载。
\end{example}

\begin{definition}
    作用可以按时间变异分为三种作用:
    \begin{enumerate}
        \item 永久作用(在结构设计基准期内不随时间变化或变化与平均值相比可忽略不计):如土压力,结构自重
        \item 可变作用(在结构设计基准期内随时间变化,且变化与平均值相比不可忽略):如风荷载,雪荷载,车辆荷载,人员荷载
        \item 偶然作用(在结构设计基准期内不一定出现,而一旦出现其量值很大且持续时间较短):如地震荷载,爆炸荷载
    \end{enumerate}
\end{definition}



\ifx\allfiles\undefined
\end{document}
\fi