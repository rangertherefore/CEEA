\ifx\allfiles\undefined
\documentclass[12pt, a4paper, oneside, UTF8]{ctexbook}
\def\path{../config}
\input{../config/_config}
\begin{document}

\else
\fi

\chapter{重力荷载}

\section{结构自重}

\begin{definition}
    构件自重:
\[ G_b = \gamma V \]
结构总自重:
\[ G = \sum_{i=1}^{n} \gamma_i V_i \]
\end{definition}

\section{土压力}

\begin{definition}
    成层土中竖向自重应力沿深度的分布:
$$
\sigma_{cz} = \gamma_1 h_1 + \gamma_2 h_2 + \cdots + \gamma_n h_n = \sum_{i=1}^{n} \gamma_i h_i
$$
如果考虑地下水位的话,存在突变。

具体为,假如到深度$h_1$处为地下水位线,由刚才土自重应力计算公式得,$\sigma_1=\gamma_1 h_1$

地下水位到$h_2$结束,那么$h_2$上端为$\gamma_1 h_1 + (\gamma_2 -\gamma_w )h_2$ (可以理解为去掉了水的浮力)

但是$h_2$下端,突变为$\gamma_1 h_1 + \gamma_2 h_2$ (这里不考虑水的重力了)
\end{definition}

\begin{figure}[H]
    \centering
    \includegraphics[width=0.5\textwidth]{../figure/tuyali}
    \caption{土压力示意图}
    \label{fig:tuyali}
\end{figure}

\section{雪荷载}

\begin{definition}
    $s = \gamma d = \rho g d$
\begin{enumerate}
    \item $s$: 雪压 ($kN/m^2$)
    \item $d$: 积雪深度,指从积雪表面到地面的垂直深度 (m)
    \item $\gamma$: 雪重度 ($kN/m^3$)
    \item $\rho$: \text{积雪密度 ($kg/m^3$)}
    \item $g$: 重力加速度,取$9.8m/s^2$。
\end{enumerate}
\label{eq:snow_pressure}
\end{definition}

\begin{remark}
    雪荷载喜欢考定义,主要影响因素是雪深和雪重度
\end{remark}

\textbf{雪密度随雪深的变化:}

刚刚飘落的雪十分蓬松,密度较小,大约为:50~100 kg/m³

积雪达到一定厚度时,下层积雪压密,雪密度增加。

\textbf{雪密度随时间的变化:}

积雪反复冻融作用及人为踩踏搅动,其密度也会增加。

\textbf{海拔高度对基本雪压的影响:}

一般基本雪压随海拔高度增加而增大。因为海拔较高的地区温度较低,降雪机会增多且积雪融化延缓。

\begin{remark}
    最大雪深与最大雪密度两者并不一定同时出现。

最好是直接量测雪压,即直接记录地面雪压值。
\end{remark}

\begin{definition}
    基本雪压:

空旷平坦地面上,根据当地气象台观察并收集的年\textbf{最大雪压},经统计得出的50年一遇的最大雪压(即重现期为50年)的概率分布(极值分布),取分布上取某个分位值(\( p_k \)=0.36)为雪压标准值;对雪敏感的结构(主要指大跨、轻质屋盖结构),应采用100年重现期的雪压分布
\end{definition}

基本雪压是针对地面上的积雪荷载定义的。屋面雪荷载由于多种因素影响,往往与地面雪荷载不同。

影响屋面积雪的因素:

风对屋面积雪的影响:漂积作用

屋面形式(坡度)对积雪的影响:雪的滑移,{\color{red}显然坡度越大,屋面雪压力越小}

屋面散热(温度)对积雪的影响:积雪融化、积雪滑移

\begin{definition}
    屋面水平投影面的雪荷载标准值按下式计算:
\[ S_k = \mu_r S_0 \]
\[ S_k \quad \text{—— 雪荷载标准值(kN/m²)} \]
\[ \mu_r \quad \text{—— 屋面积雪分布系数} \]
\[ S_0 \quad \text{—— 地面基本雪压(kN/m²)} \]
\end{definition}

\section{汽车荷载}

\begin{definition}
    汽车荷载分为车道荷载(包含均布荷载和集中荷载)和车辆荷载
\end{definition}

\begin{remark}
    车辆荷载和车道荷载的作用不得叠加
\end{remark}

\begin{figure}[H]
    \centering
    \includegraphics[width=0.8\textwidth]{../figure/qichehezai.png}
    \caption{汽车荷载等级}
    \label{fig:qichehezai}
\end{figure}

\begin{enumerate}
    \item 车辆荷载:考虑车的尺寸及车的排列方式,以集中荷载的形式作用于车轴(即车轮)位置。
    \item 车道荷载:不考虑车的尺寸及排列方式,将其等效为均布荷载和一个可作用于任意位置的集中荷载形式。
    \item 公路Ⅰ级和公路Ⅱ级汽车荷载采用相同的车辆荷载标准值。
    \item 车道荷载是个虚拟荷载,由均布荷载和集中荷载组成。
    \item 对于连续梁,常用影响线辅助寻找产生最不利效应的最不利荷载分布。
    \item 多车道桥梁上的汽车荷载应考虑多车道折减(横向折减)
    \item 大跨径桥梁随着桥梁跨度的增加桥梁上实际通行的车辆达到较高
    密度和满载的概率减小,应考虑纵向计算跨径折减。当为多跨连续结构时,整个结构应按最大的计算
    跨径考虑汽车荷载效应的折减。
\end{enumerate}

\section{楼面活荷载}

\begin{definition}
    活荷载的定义:

指建筑物中的人群、家具、设施等产生的重力作用,这些荷载的量值随时间发生变化,位置也是可移动的,亦称{\color{red}可变荷载}。

楼面活荷载按其随时间变异的特点,可分为持久性和临时性两部分:

\begin{enumerate}
    \item 持久性活荷载:(Sustained live load)是指楼面上在某个时段内基本保持不变的荷载,例如住宅内的家具、常住人员等;
    \item 临时性活荷载:(Transient live load)是指楼面上偶尔出现的短期荷载,例如聚会的人群、装修材料的堆积等。
\end{enumerate}

楼面活荷载一般处理为均布荷载,不同场景的均布荷载计算不同
\end{definition}

\begin{example}
下面结构受到的作用,属于结构自重的是:(B)  

A. 家具的重力;  

B. 楼面板受到的重力;  

C. 人员的重力;  

D. 屋顶积雪的重力。
\end{example}

\begin{remark}
    由我们刚才的讨论可以知道,家具和人员都属于楼面活荷载,设备也是。
\end{remark}

\begin{example}
    计算楼面活荷载时,为什么当活荷载作用面积超过一定数值时,需对楼面活荷载进行折减?

    由于楼面均布活荷载可理解为总活荷载按楼面面积平均,面积越大,平摊的楼面活荷载越小。
\end{example}

\section{人群荷载}

\begin{definition}
    人群荷载分为公路桥梁人群荷载、城市桥梁人群荷载、铁路桥梁人行道荷载

    (暂时不知道啥是重点)
\end{definition}

\ifx\allfiles\undefined
\end{document}
\fi