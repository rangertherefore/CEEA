\ifx\allfiles\undefined
\documentclass[12pt, a4paper, oneside, UTF8]{ctexbook}
\def\path{../config}
\input{../config/_config}
\begin{document}
% \input{../config/cover}
\else
\fi

\chapter{无机气硬性胶凝材料}

\section{石灰}

1.石灰的生产:主要是高温煅烧石灰石(碳酸钙)。煅烧温度过低或时间不足,生成欠火石灰,降低有效利用率;反之生成过火石灰,与水反应熟化速度慢,产生局部体积膨胀,影响工程质量。

为消除过火石灰危害,石灰膏使用前应在化灰池中存放2周以上,使过火石灰充分熟化,称为“陈伏”。

2.石灰的熟化:生石灰氧化钙加水反应生成 Ca(OH)₂ 的过程。

特点:(1)速度快(2)体积膨胀(3)放出大量的热

3. 石灰的凝结硬化:Ca(OH)₂ 和空气中的 CO₂ 和水反应,形成不溶于水的碳酸钙晶体,析出的水分则逐渐被蒸发。碳化反应长期局限于表层,速度慢。
$$Ca(OH)_2 + CO_2 + nH_2O = CaCO_3 + (n+1)H_2O$$
特点:(1)速度慢(2)体积收缩大

\begin{remark}
    石灰的技术性质
    \begin{enumerate}
        \item 可塑性好:石灰浆里的氢氧化钙颗粒小,分散均匀,浆体流动性好,能填充细小的缝隙。
        \item 耐水性差:氢氧化钙在水中溶解度大
        \item 硬化体积收缩:原本胶体里的毛细管由于水分蒸发,被破坏
        \item 硬化缓慢:长期发生在表层,且空气中二氧化碳浓度低
        \item 硬化强度低:硬化在表面,且硬化密度低。
        \item 保水性:石灰具有较强的吸水能力,尤其是在与水反应后,能够保持较长时间的湿润状态。这使得它能有效地保持水分。
    \end{enumerate}
\end{remark}

\section{石膏}

生产建筑石膏的原料主要有天然二水石膏,或含有二水石膏或含有二水石膏与无水石膏混合物的化工副产品及废渣(如磷石膏是制造磷酸时的废渣,此外还有氟石膏、盐石膏、硼石膏、黄石膏、钛石膏、脱硫石膏等)。

石膏泛指软石膏和硬石膏两种矿物。软石膏为二水硫酸钙(\(\mathrm{CaSO_{4}\cdot2H_{2}O}\)),又称二水石膏;硬石膏为无水硫酸钙(\(\mathrm{CaSO_{4}}\))。

石膏的凝结硬化:(1)水化:$CaSO_4 \cdot \frac{1}{2}H_2O + \frac{3}{2}H_2O \rightleftharpoons CaSO_4 \cdot 2H_2O$

由于二水石膏的溶解度比半水石膏小,故二水石膏首先从饱和溶液中析晶沉淀。

(2)凝结硬化:二水石膏颗粒比半水石膏细,比表面积大,可以吸附更多的水,从而使石膏浆体很快失去塑性而凝结。(孔隙率大)

\begin{remark}
    对比石灰的凝结硬化,石灰凝结硬化速度慢,且强度低,而且由于石灰浆体的孔隙率大,水分蒸发后,石灰浆体会收缩。

    但是石膏凝结速度快,且强度高(相对于石灰),石膏浆体的孔隙率小,水分蒸发后,石膏浆体不会收缩。
\end{remark}

\textbf{石膏的技术性质:}

\begin{enumerate}
    \item 凝结硬化快
    \item 强度较高(与石灰比)
    \item 体积微膨胀
    \item 色白可加彩色
    \item 保温性能好
    \item 耐水性差但具有一定调湿性能
    \item 防火性好
\end{enumerate}

\newpage

\section{水玻璃}
水玻璃的组成:常用的为硅酸钠(Na2O • nSiO2)水溶液,又称钠水玻璃。

其中 n 称为水玻璃的模数,n 越大,水玻璃黏性和强度越高,但在水中溶解能力下降。土木工程常用模数 n 为 2.6-2.8 的水玻璃,我国生产一般为 2.4-3.3。

水玻璃的凝结固化:

2[Na2O • nSiO2] + Na2SiF6 + mH2O = (2n+1)SiO2 • mH2O + 6NaF,

其中氟硅酸钠为固化剂,目的是加快凝结固化速度和提高强度。

\begin{example}
    水玻璃的模数越高越好吗?

    虽然模数提高,可以加快水玻璃的凝结速度和强度,但是模数过高,不利于施工操作,影响其使用效果。
\end{example}

\textbf{水玻璃技术性质:}
\begin{enumerate}
    \item 黏结力和强度较高
    \item 耐酸性好,只与氢氟酸反应\
    \item 耐热性好
    \item 耐碱性和耐水性差,但可采用中等浓度酸对已硬化的水玻璃酸洗,提高耐水性。
    \item 抗渗透性好
\end{enumerate}

镁质胶凝材料:菱苦土(其中之一),生产:

\[
\text{MgCO}_3 \xrightarrow{750 \sim 800^\circ \text{C}} \text{MgO} + \text{CO}_2
\]

\section{镁质胶凝材料}

与石灰类似,也有欠火和过火问题。

凝结硬化:应通过盐溶液拌合,如 MgCl2、MgSO4、FeCl3、FeSO4 等,采用 MgCl2 拌合的硬化后强度可达 40~60MPa,称为氯氧镁水泥。

镁质胶凝材料的技术性质:

\begin{enumerate}
    \item 具有一定导电性,可防止静电积聚
    \item 耐水性差
    \item 黏结性能好
    \item 强度高,凝结时间可调
\end{enumerate}

\section{材料性质和用途对比}

\begin{table}[ht]
    \centering
    \begin{tabular}{|c|c|c|c|c|}
        \hline
        \textbf{材料} & \textbf{凝结时间} & \textbf{强度} & \textbf{耐水性} & \textbf{用途} \\ \hline
        石灰 & 慢 & 低 & 差 & 室内粉刷,无声爆破剂 \\ \hline
        石膏 & 快 & 中等 & 差 & 粉刷石膏,作为内墙涂料,或作为混凝土添加剂 \\ \hline
        水玻璃 & 快 & 高 & 差 & 粘结剂、涂料 \\ \hline
        镁质胶凝材料 & 中等 & 高 & 差 & 用做外包装材料,制作地板(防静电) \\ \hline
    \end{tabular}
    \caption{无机气硬性胶凝材料的性质和用途对比}
\end{table}

\begin{remark}
    石灰和石膏都是属于耐水性差,保水性好的材料
\end{remark}

\section{作业题}

\begin{example}
    某多层住宅楼室内抹灰采用的是石灰砂浆,交付使用后出现墙面普遍鼓包开
裂,试分析其原因。欲避免这种情况发生,应采取什么措施? 

    石灰砂浆里含过火石灰未反应,导致延后反应生成熟石灰,鼓包开裂。
\end{example}

\begin{remark}
    注意对比,过火石灰导致体积膨胀开裂,失水硬化导致体积收缩。
\end{remark}

\begin{example}
    石灰是气硬性胶凝材料,为什么由它配制的石灰土和三合土可以用来建造灰
土渠道、 三合土滚水坝等水工建筑物?

    石灰保水性好,具有调节水分的功能。配置三合土和灰土之后,三合土和灰土在强力夯打之下,密实度大大提高。
粘土中的少量活性氧化硅和氧化铝可与石灰粉水化产物Ca(OH)2发生作业,生成水硬性矿物,因此具有一定抗压强度、耐水性和良好的抗渗能力。因而具有一定抗压强度、耐水性和相当高的抗渗能力。
\end{example}

\begin{example}
    室外相比室内湿度的不确定性更大,而二水石膏由于表面积大,吸湿能力强,石膏饱水后强度大幅下降。
\end{example}

\begin{example}
    用于内墙抹灰时,与石灰相比,建筑石膏具有哪些优点?

    体积微膨胀,凝结硬化快,强度相对高。
\end{example}

\begin{example}
石灰硬化过程中会产生哪几种开裂破坏?欲避免这些开裂发生,应采取什么技术措施?

墙面上出现的不规则网状裂纹是石灰在凝结硬化中产生较大收缩而引起的。可以通过增加砂用量,掺入纤维材料、湿润墙体基层、降低一次抹灰的厚度等措施加以改善。

墙面上出现凸出的呈放射状的裂纹是由于石灰中含有过火石灰。在砂浆硬化后,过火石灰吸收空气中的水蒸气继续熟化,体积膨胀,从而出现上述现象。为避免这种情况发生,可在熟化时先用筛网将较大的过火石灰颗粒过滤掉,较小的过火石灰颗粒可通过陈伏使其熟化。
\end{example}

\begin{example}
用于内墙抹灰时,与石灰相比,建筑石膏具有那些优点?为什么?

1、凝结硬化快

2、色白可加色彩,建筑石膏颜色洁白。杂质含量越少,颜色越白。可加入各种颜料调制成彩色石膏制品,且保色性能好。

3、体积微膨胀,建筑石膏凝结硬化过程的体积微膨胀特性,使得石膏制品表面光滑、提醒饱满、无收缩裂纹,特别是用于刷面和制作建筑装饰制品。而石灰干燥收缩大。

4、保温性能好。由于石膏制品生产时往往加入过量的水,蒸发后形成大量的内部毛细孔,孔隙率达50-60\%,表观密度小(800-1000kg/m3),导热系数小,故具有良好的保温绝热性能,常用作保温隔热材料,并具有一定的吸声效果。
\end{example}

\begin{example}
    水玻璃为什么不能涂在石膏上?

    水玻璃溶液强碱(pH>11),会与石膏中的CaSO₄·2H₂O发生碱性分解反应:
    $$CaSO_4 + 2NaOH \longrightarrow Ca(OH)_2 \downarrow + Na_2SO_4$$
    反应生成的$Ca(OH)_2$不牢固,$Na_2SO_4$又是可溶盐,会析出结晶并膨胀、脱落,破坏石膏基面。

    同时,石膏本身微溶于水,吸水后表面会软化甚至微膨胀,导致水玻璃膜层不能紧密贴合,出现龟裂、起皮。
\end{example}

\ifx\allfiles\undefined
\end{document}
\fi