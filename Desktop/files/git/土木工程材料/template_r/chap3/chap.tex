\ifx\allfiles\undefined
\documentclass[12pt, a4paper, oneside, UTF8]{ctexbook}
\def\path{../config}
\usepackage{amsmath}
\usepackage{amsthm}
\usepackage{array}
\usepackage{amssymb}
\usepackage{graphicx}
\usepackage{mathrsfs}
\usepackage{enumitem}
\usepackage{geometry}
\usepackage[colorlinks, linkcolor=black]{hyperref}
\usepackage{stackengine}
\usepackage{yhmath}
\usepackage{extarrows}
% \usepackage{unicode-math}
\usepackage{esint}
\usepackage{multirow}
\usepackage{fancyhdr}
\usepackage[dvipsnames, svgnames]{xcolor}
\usepackage{listings}
\usepackage{float} % Required for the H float option
\definecolor{mygreen}{rgb}{0,0.6,0}
\definecolor{mygray}{rgb}{0.5,0.5,0.5}
\definecolor{mymauve}{rgb}{0.58,0,0.82}
\definecolor{NavyBlue}{RGB}{0,0,128}
\definecolor{Rhodamine}{RGB}{255,0,255}
\definecolor{PineGreen}{RGB}{0,128,0}

\graphicspath{ {figures/},{../figures/}, {config/}, {../config/} }

\linespread{1.6}

\geometry{
    top=25.4mm, 
    bottom=25.4mm, 
    left=20mm, 
    right=20mm, 
    headheight=2.17cm, 
    headsep=4mm, 
    footskip=12mm
}

\setenumerate[1]{itemsep=5pt,partopsep=0pt,parsep=\parskip,topsep=5pt}
\setitemize[1]{itemsep=5pt,partopsep=0pt,parsep=\parskip,topsep=5pt}
\setdescription{itemsep=5pt,partopsep=0pt,parsep=\parskip,topsep=5pt}

\lstset{
    language=Mathematica,
    basicstyle=\tt,
    breaklines=true,
    keywordstyle=\bfseries\color{NavyBlue}, 
    emphstyle=\bfseries\color{Rhodamine},
    commentstyle=\itshape\color{black!50!white}, 
    stringstyle=\bfseries\color{PineGreen!90!black},
    columns=flexible,
    numbers=left,
    numberstyle=\footnotesize,
    frame=tb,
    breakatwhitespace=false,
} 

\lstset{
    language=TeX, % 设置语言为 TeX
    basicstyle=\ttfamily, % 使用等宽字体
    breaklines=true, % 自动换行
    keywordstyle=\bfseries\color{NavyBlue}, % 关键字样式
    emphstyle=\bfseries\color{Rhodamine}, % 强调样式
    commentstyle=\itshape\color{black!50!white}, % 注释样式
    stringstyle=\bfseries\color{PineGreen!90!black}, % 字符串样式
    columns=flexible, % 列的灵活性
    numbers=left, % 行号在左侧
    numberstyle=\footnotesize, % 行号字体大小
    frame=tb, % 顶部和底部边框
    breakatwhitespace=false % 不在空白处断行
}

% \begin{lstlisting}[language=TeX] ... \end{lstlisting}

% 定理环境设置
\usepackage[strict]{changepage} 
\usepackage{framed}

\definecolor{greenshade}{rgb}{0.90,1,0.92}
\definecolor{redshade}{rgb}{1.00,0.88,0.88}
\definecolor{brownshade}{rgb}{0.99,0.95,0.9}
\definecolor{lilacshade}{rgb}{0.95,0.93,0.98}
\definecolor{orangeshade}{rgb}{1.00,0.88,0.82}
\definecolor{lightblueshade}{rgb}{0.8,0.92,1}
\definecolor{purple}{rgb}{0.81,0.85,1}

\theoremstyle{definition}
\newtheorem{myDefn}{\indent Definition}[section]
\newtheorem{myLemma}{\indent Lemma}[section]
\newtheorem{myThm}[myLemma]{\indent Theorem}
\newtheorem{myCorollary}[myLemma]{\indent Corollary}
\newtheorem{myCriterion}[myLemma]{\indent Criterion}
\newtheorem*{myRemark}{\indent Remark}
\newtheorem{myProposition}{\indent Proposition}[section]

\newenvironment{formal}[2][]{%
	\def\FrameCommand{%
		\hspace{1pt}%
		{\color{#1}\vrule width 2pt}%
		{\color{#2}\vrule width 4pt}%
		\colorbox{#2}%
	}%
	\MakeFramed{\advance\hsize-\width\FrameRestore}%
	\noindent\hspace{-4.55pt}%
	\begin{adjustwidth}{}{7pt}\vspace{2pt}\vspace{2pt}}{%
		\vspace{2pt}\end{adjustwidth}\endMakeFramed%
}

\newenvironment{definition}{\vspace{-\baselineskip * 2 / 3}%
	\begin{formal}[Green]{greenshade}\vspace{-\baselineskip * 4 / 5}\begin{myDefn}}
	{\end{myDefn}\end{formal}\vspace{-\baselineskip * 2 / 3}}

\newenvironment{theorem}{\vspace{-\baselineskip * 2 / 3}%
	\begin{formal}[LightSkyBlue]{lightblueshade}\vspace{-\baselineskip * 4 / 5}\begin{myThm}}%
	{\end{myThm}\end{formal}\vspace{-\baselineskip * 2 / 3}}

\newenvironment{lemma}{\vspace{-\baselineskip * 2 / 3}%
	\begin{formal}[Plum]{lilacshade}\vspace{-\baselineskip * 4 / 5}\begin{myLemma}}%
	{\end{myLemma}\end{formal}\vspace{-\baselineskip * 2 / 3}}

\newenvironment{corollary}{\vspace{-\baselineskip * 2 / 3}%
	\begin{formal}[BurlyWood]{brownshade}\vspace{-\baselineskip * 4 / 5}\begin{myCorollary}}%
	{\end{myCorollary}\end{formal}\vspace{-\baselineskip * 2 / 3}}

\newenvironment{criterion}{\vspace{-\baselineskip * 2 / 3}%
	\begin{formal}[DarkOrange]{orangeshade}\vspace{-\baselineskip * 4 / 5}\begin{myCriterion}}%
	{\end{myCriterion}\end{formal}\vspace{-\baselineskip * 2 / 3}}
	

\newenvironment{remark}{\vspace{-\baselineskip * 2 / 3}%
	\begin{formal}[LightCoral]{redshade}\vspace{-\baselineskip * 4 / 5}\begin{myRemark}}%
	{\end{myRemark}\end{formal}\vspace{-\baselineskip * 2 / 3}}

\newenvironment{proposition}{\vspace{-\baselineskip * 2 / 3}%
	\begin{formal}[RoyalPurple]{purple}\vspace{-\baselineskip * 4 / 5}\begin{myProposition}}%
	{\end{myProposition}\end{formal}\vspace{-\baselineskip * 2 / 3}}


\newtheorem{example}{\indent \color{SeaGreen}{Example}}[section]
\renewcommand{\proofname}{\indent\textbf{\textcolor{TealBlue}{Proof}}}
\newenvironment{solution}{\begin{proof}[\indent\textbf{\textcolor{TealBlue}{Solution}}]}{\end{proof}}

% 自定义命令的文件

\def\d{\mathrm{d}}
\def\R{\mathbb{R}}
%\newcommand{\bs}[1]{\boldsymbol{#1}}
%\newcommand{\ora}[1]{\overrightarrow{#1}}
\newcommand{\myspace}[1]{\par\vspace{#1\baselineskip}}
\newcommand{\xrowht}[2][0]{\addstackgap[.5\dimexpr#2\relax]{\vphantom{#1}}}
\newenvironment{mycases}[1][1]{\linespread{#1} \selectfont \begin{cases}}{\end{cases}}
\newenvironment{myvmatrix}[1][1]{\linespread{#1} \selectfont \begin{vmatrix}}{\end{vmatrix}}
\newcommand{\tabincell}[2]{\begin{tabular}{@{}#1@{}}#2\end{tabular}}
\newcommand{\pll}{\kern 0.56em/\kern -0.8em /\kern 0.56em}
\newcommand{\dive}[1][F]{\mathrm{div}\;\boldsymbol{#1}}
\newcommand{\rotn}[1][A]{\mathrm{rot}\;\boldsymbol{#1}}

% 修改参数改变封面样式,0 默认原始封面、内置其他1、2、3种封面样式
\def\myIndex{0}


\ifnum\myIndex>0
    \input{\path/cover_package_\myIndex}
\fi

\def\myTitle{标题:一份LaTeX笔记模板}
\def\myAuthor{作者名称}
\def\myDateCover{封面日期: \today}
\def\myDateForeword{前言页显示日期: \today}
\def\myForeword{前言标题}
\def\myForewordText{
    
    这是一个基于\LaTeX{}的模板,用于撰写学习笔记。

    模板旨在提供一个简单、易用的框架,以便你能够专注于内容,而不是排版细节,如不是专业者,不建议使用者在模板细节上花费太多时间,而是直接使用模板进行笔记撰写。遇到问题,再进行调整解决。
}
\def\mySubheading{副标题}


\begin{document}
% \input{\path/cover_text_\myIndex.tex}

\newpage
\thispagestyle{empty}
\begin{center}
    \Huge\textbf{\myForeword}
\end{center}
\myForewordText
\begin{flushright}
    \begin{tabular}{c}
        \myDateForeword
    \end{tabular}
\end{flushright}

\newpage
\pagestyle{plain}
\setcounter{page}{1}
\pagenumbering{Roman}
\tableofcontents

\newpage
\pagenumbering{arabic}
\setcounter{chapter}{-1}
\setcounter{page}{1}

\pagestyle{fancy}
\fancyfoot[C]{\thepage}
\renewcommand{\headrulewidth}{0.4pt}
\renewcommand{\footrulewidth}{0pt}








\else
\fi

\chapter{材料基本性质}
\section{基本定义}

\begin{definition}
    四种密度
    \begin{enumerate}
        \item 密度$\rho$:绝对密实情况。碾成粉然后采用排水法。
        \item 视密度$\rho_a$:仅包含闭口孔隙
        \item 表观密度$\rho_o$:考虑开口和闭口孔隙。
        \item 堆积密度$\rho_o^{\prime}$:考虑开闭口孔隙,同时考虑空隙。
    \end{enumerate}
    显然:
    \[
    \rho>\rho_a>\rho_o>\rho_o^{\prime}
    \]
\end{definition}

密实度是材料体积内被固体物质充实的程度。按下式计算:
$$
D = \frac{V}{V_0} \times 100\% \quad \text{或} \quad D = \frac{\rho_0}{\rho} \times 100\%
$$
孔隙率是材料体积内,孔隙体积所占的比例。按下式计算:
$$
P = \frac{V_0 - V}{V_0} = 1 - \frac{V}{V_0} = \left(1 - \frac{\rho_0}{\rho}\right) \times 100\%
$$

\begin{remark}
    D+P=1或 密实度+孔隙率=1
\end{remark}

\begin{definition}
    耐水性

材料抵抗水破坏作用的性质称为耐水性,用软化系数表示,即
\[ K_R = \frac{f_b}{f_g} \]
\end{definition}

\begin{definition}
    抗渗性:材料抵抗压力液体渗透的能力

当材料两侧存在不同压力时,一切破坏因素(如腐蚀性介质)都可通过水或气体进入材料内部,然后把所分解的产物代出材料,使材料逐渐破坏,如地下建筑、基础、压力管道、水工建筑等经常受到压力水或水头差的作用,故要求所用材料具有一定的抗渗性,对于各种防水材料,则要求具有更高的抗渗性。
\end{definition}

\begin{definition}
    渗透系数的物理意义是:在一定时间t内,透过材料试件的水量Q,与试件的渗水面积A及水头差成正比,与渗透距离(试件的厚度) d成反比,用公式表示为

\[ K = \frac{Qd}{AtH} \]
\end{definition}

\begin{definition}
    S=10H-1

式中:S---抗渗等级;
H ---试件开始渗水时的压力(MPa)
\end{definition}

\section{作业题}

\begin{example}
    考虑孔隙率变化,材料性质会如何改变,$P_o$是开口孔隙率,$P_c$是闭口孔隙率({\color{red}第二、三行孔隙率不变,只改变开口或者闭口孔隙率})
    \begin{table}[h!]
        \centering
        \begin{tabular}{|c|c|c|c|c|c|c|c|c|c|}
        \hline
        孔隙率 & 密度 & 表观密度 & 强度 & 吸水性 & 吸湿性 & 抗冻性 & 抗渗性 & 导热性 & 吸声性 \\
        \hline
        P $\uparrow$ & = & $\downarrow$ & $\downarrow$ & $\uparrow$ & $\uparrow$ & $\downarrow$ & $\downarrow$ & $\downarrow$ & $\uparrow$ \\
        \hline
        P$_o$ $\uparrow$ & = & = & $\downarrow$ & $\uparrow$ & $\uparrow$ & $\downarrow$ & $\downarrow$ & $\uparrow$ & $\uparrow$ \\
        \hline
        P$_c$ $\uparrow$ & = & = & $\uparrow$ & $\downarrow$ & $\downarrow$ & $\uparrow$ & $\uparrow$ & $\downarrow$ & $\downarrow$ \\
        \hline
        \end{tabular}
        \caption{孔隙率与其他物理性质变化的关系表}
    \end{table}
\end{example}

\begin{remark}
    开口孔隙会降低强度,但是微闭口孔隙会提高强度。

    表观密度包括开口也包括闭口孔隙,所以不变。

    开口孔隙会降低抗冻性。
\end{remark}

\begin{example}
    1-2 今有湿砂100.0kg,已知其含水率为6.0\%,则其中含有的水的质量是多少?
$$
\frac{m_{ssd} - m}{m} \times 100\% = 6.0\%
$$
其中 $m_{ssd} = 100.0$kg
$$
m = 94.3 \text{kg}
$$
$$
m_w = m \times W_m = 5.66 \text{kg}
$$
\end{example}

\begin{example}
某石材在气干、绝干、水饱和条件下测得的抗压强度分别为74.0、80.0、65.0 MPa,求该石材的软化系数,并判断该石材可否用于水下工程。P9
$$
K_R = \frac{f_b}{f_g} = \frac{65}{80} = 0.81 < 0.85
$$
不可用于水下工程
\end{example}

\begin{remark}
    用绝干来计算
\end{remark}

\begin{example}
称取堆积密度为1480kg/m³的干砂300g,将此砂加入已装有250mL水的500mL容量瓶内,充分摇动排尽气泡,静止24h小时后加水到刻度,称得总重量为876g;将瓶内砂和水倒出洗净后,再向瓶内重新注水到刻度,此时称得总重量为690 g。再将砂敲碎磨细过筛(0.2 mm)烘干后,取样53.81 g,测得其排水体积V排水为19.85 cm³。试计算该砂的空隙率。
$$
\text{空隙率计算公式为:}P=\left(1 - \frac{\rho_0^{\prime}}{\rho}\right) \times 100\%,\text{我们需要知道堆积密度和绝对密度}
$$
$$
\rho = \frac{m}{V} = \frac{53.81}{19.85} = 2.71\,\mathrm{g}/\mathrm{cm}^3
$$
$$
\rho_0 = 300 \times \frac{1}{300 + 690 - 876} = 2.63\text{g}/\text{cm}^3
$$
$$
P = \left(1 - \frac{\rho_0^{\prime}}{\rho}\right) \times 100\% = \left(1 - \frac{2.63}{2.71}\right) \times 100\% = 43.73\%
$$
\end{example}

\begin{example}
一块砖块,其真实密度 $\rho = 2.60 \, \text{g/cm}^3$,绝干状态下的质量 $m_0 = 2000 \, \text{g}$,将它浸没于水中,吸水饱和后取出表面,测得其饱和质量 $m_h = 2350 \, \text{g}$;另外用静水天平测得其浸泡于水中的质量 $m_2 = 1150 \, \text{g}$。求该砖的表观密度、体积吸水率、孔隙率、开口孔隙率和闭口孔隙率。

\begin{remark}
    静水天平的测量原理:物体在空气中的重力和在水中产生的荷载的差,为浮力,可以同时测量质量和体积。
\end{remark}

表观密度(实际上对应浙大版的视密度,只算了开口孔): 
$$
\rho_o = \frac{m_0}{m_0 - m_2} = \frac{2000}{2000 - 1150} = 2.35 \, \text{g/cm}^3
$$

体积密度(实际上对应浙大版的表观密度,因为开口和闭口孔都算上了): 
$$
\rho_\omega = \frac{m_0}{m_1 - m_2} = \frac{2000}{2350 - 1150} = 1.67 \, \text{g/cm}^3
$$

孔隙率:
$$
P = \left( 1 - \frac{\rho_\omega}{\rho_t} \right) \times 100\% = \left( 1 - \frac{1.67}{2.60} \right) \times 100\% = 35.8\%
$$

体积吸水率(只算开口孔):
$$
W_v = P = \left( 1 - \frac{\rho_\omega}{\rho_e} \right) \times 100\% = \left( 1 - \frac{1.67}{2.35} \right) \times 100\% = 28.9\%
$$

开口孔隙率:
$$
P_o = W_v = 28.9\%
$$

闭口孔隙率:
$$
P_c = P - P_o = 35.8\% - 28.9\% = 6.9\%
$$
\end{example}

\begin{example}
    称取堆积密度为1480kg/m³的干砂300g,将此砂加入已装有250mL水的500mL容量瓶内,充分摇动排尽气泡,静止24h小时后加水到刻度,称得总重量为876g;将瓶内砂和水倒出洗净后,再向瓶内重新注水到刻度,此时称得总重量为690 g。再将砂敲碎磨细过筛(0.2 mm)烘干后,取样53.81 g,测得其排水体积V排水为19.85 cm³。试计算该砂的空隙率。

    首先可以轻松算出绝对密度:
    \[
    \rho = \frac{m}{V} = \frac{53.81}{19.85} = 2.71\,\mathrm{g}/\mathrm{cm}^3
    \]
    其次,这里(876-690)g是砂比同体积的水重的质量,所以砂的体积就是300-(876-690)g的水的体积: 
    
    因此表观密度就是(这里你可能疑惑为什么开口闭口啥都没说,为啥就是表观密度了?但是就是,我也不知道为啥,查规范查的):
    \[
    \rho_0 = 300 \times \frac{1}{300 + 690 - 876} = 2.63\text{g}/\text{cm}^3
    \]
    因此,孔隙率就是:
    \[
    P = \left(1 - \frac{\rho_0^{\prime}}{\rho}\right) \times 100\% = \left(1 - \frac{2.63}{2.71}\right) \times 100\% = 43.73\%
    \]
\end{example}

\begin{example}
    评价材料耐水性的指标是什么?

    软化系数:$K_R=\frac{f_b}{f_g}$,大于0.85为耐水材料,类似的有渗透系数,描述渗水能力,$K=\frac{Qd}{AtH}$(H是两侧压强的水头差),$S=10H-1$,渗透等级(H —试件开始渗水时的压力(MPa))。
\end{example}

\begin{example}
    缩写解释题,历年卷有考过的几个高频考点:F50 C25 M7.5 Q235 MU20 42.5R什么意思?

    抗冻能力,用符号Fn表示,其中n即为最大冻融循环次数。 例如F25、F50等。类似需要复习的符号有C25(C是混凝土符号,25是指28天标准抗压强度为25MPa),M7.5(砂浆28天标准养护的抗压强度),42.5(水泥28天标准养护的抗压强度,当然还要比较抗折强度,带R的是早强的),Q是钢材的屈服强度,大于235MPa,然后MU 是砖块的抗压等级,大于20MPa。。
\end{example}

\begin{example}
    有没有导热系数比空气还小的材料?

气凝胶材料由纳米级孔道和极低密度结构组成,极大程度抑制了气体和固体的热传导。
\end{example}


\ifx\allfiles\undefined
\end{document}
\fi