\ifx\allfiles\undefined
\documentclass[12pt, a4paper, oneside, UTF8]{ctexbook}
\def\path{../config}
\input{../config/_config}
\begin{document}
% \input{../config/cover}
\else
\fi

\chapter{材料基本性质}
\section{基本定义}

\begin{definition}
    四种密度
    \begin{enumerate}
        \item 密度$\rho$:绝对密实情况。碾成粉然后采用排水法。
        \item 视密度$\rho_a$:仅包含闭口孔隙
        \item 表观密度$\rho_o$:考虑开口和闭口孔隙。
        \item 堆积密度$\rho_o^{\prime}$:考虑开闭口孔隙,同时考虑空隙。
    \end{enumerate}
    显然:
    \[
    \rho>\rho_a>\rho_o>\rho_o^{\prime}
    \]
\end{definition}

密实度是材料体积内被固体物质充实的程度。按下式计算:
$$
D = \frac{V}{V_0} \times 100\% \quad \text{或} \quad D = \frac{\rho_0}{\rho} \times 100\%
$$
孔隙率是材料体积内,孔隙体积所占的比例。按下式计算:
$$
P = \frac{V_0 - V}{V_0} = 1 - \frac{V}{V_0} = \left(1 - \frac{\rho_0}{\rho}\right) \times 100\%
$$

\begin{remark}
    D+P=1或 密实度+孔隙率=1
\end{remark}

\begin{definition}
    耐水性

材料抵抗水破坏作用的性质称为耐水性,用软化系数表示,即
\[ K_R = \frac{f_b}{f_g} \]
\end{definition}

\begin{definition}
    抗渗性:材料抵抗压力液体渗透的能力

当材料两侧存在不同压力时,一切破坏因素(如腐蚀性介质)都可通过水或气体进入材料内部,然后把所分解的产物代出材料,使材料逐渐破坏,如地下建筑、基础、压力管道、水工建筑等经常受到压力水或水头差的作用,故要求所用材料具有一定的抗渗性,对于各种防水材料,则要求具有更高的抗渗性。
\end{definition}

\begin{definition}
    渗透系数的物理意义是:在一定时间t内,透过材料试件的水量Q,与试件的渗水面积A及水头差成正比,与渗透距离(试件的厚度) d成反比,用公式表示为

\[ K = \frac{Qd}{AtH} \]
\end{definition}

\begin{definition}
    S=10H-1

式中:S---抗渗等级;
H ---试件开始渗水时的压力(MPa)
\end{definition}

\section{作业题}

\begin{example}
    考虑孔隙率变化,材料性质会如何改变,$P_o$是开口孔隙率,$P_c$是闭口孔隙率({\color{red}第二、三行孔隙率不变,只改变开口或者闭口孔隙率})
    \begin{table}[h!]
        \centering
        \begin{tabular}{|c|c|c|c|c|c|c|c|c|c|}
        \hline
        孔隙率 & 密度 & 表观密度 & 强度 & 吸水性 & 吸湿性 & 抗冻性 & 抗渗性 & 导热性 & 吸声性 \\
        \hline
        P $\uparrow$ & = & $\downarrow$ & $\downarrow$ & $\uparrow$ & $\uparrow$ & $\downarrow$ & $\downarrow$ & $\downarrow$ & $\uparrow$ \\
        \hline
        P$_o$ $\uparrow$ & = & = & $\downarrow$ & $\uparrow$ & $\uparrow$ & $\downarrow$ & $\downarrow$ & $\uparrow$ & $\uparrow$ \\
        \hline
        P$_c$ $\uparrow$ & = & = & $\uparrow$ & $\downarrow$ & $\downarrow$ & $\uparrow$ & $\uparrow$ & $\downarrow$ & $\downarrow$ \\
        \hline
        \end{tabular}
        \caption{孔隙率与其他物理性质变化的关系表}
    \end{table}
\end{example}

\begin{remark}
    开口孔隙会降低强度,但是微闭口孔隙会提高强度。

    表观密度包括开口也包括闭口孔隙,所以不变。

    开口孔隙会降低抗冻性。
\end{remark}

\begin{example}
    1-2 今有湿砂100.0kg,已知其含水率为6.0\%,则其中含有的水的质量是多少?
$$
\frac{m_{ssd} - m}{m} \times 100\% = 6.0\%
$$
其中 $m_{ssd} = 100.0$kg
$$
m = 94.3 \text{kg}
$$
$$
m_w = m \times W_m = 5.66 \text{kg}
$$
\end{example}

\begin{example}
某石材在气干、绝干、水饱和条件下测得的抗压强度分别为74.0、80.0、65.0 MPa,求该石材的软化系数,并判断该石材可否用于水下工程。P9
$$
K_R = \frac{f_b}{f_g} = \frac{65}{80} = 0.81 < 0.85
$$
不可用于水下工程
\end{example}

\begin{remark}
    用绝干来计算
\end{remark}

\begin{example}
称取堆积密度为1480kg/m³的干砂300g,将此砂加入已装有250mL水的500mL容量瓶内,充分摇动排尽气泡,静止24h小时后加水到刻度,称得总重量为876g;将瓶内砂和水倒出洗净后,再向瓶内重新注水到刻度,此时称得总重量为690 g。再将砂敲碎磨细过筛(0.2 mm)烘干后,取样53.81 g,测得其排水体积V排水为19.85 cm³。试计算该砂的空隙率。
$$
\text{空隙率计算公式为:}P=\left(1 - \frac{\rho_0^{\prime}}{\rho}\right) \times 100\%,\text{我们需要知道堆积密度和绝对密度}
$$
$$
\rho = \frac{m}{V} = \frac{53.81}{19.85} = 2.71\,\mathrm{g}/\mathrm{cm}^3
$$
$$
\rho_0 = 300 \times \frac{1}{300 + 690 - 876} = 2.63\text{g}/\text{cm}^3
$$
$$
P = \left(1 - \frac{\rho_0^{\prime}}{\rho}\right) \times 100\% = \left(1 - \frac{2.63}{2.71}\right) \times 100\% = 43.73\%
$$
\end{example}

\begin{example}
一块砖块,其真实密度 $\rho = 2.60 \, \text{g/cm}^3$,绝干状态下的质量 $m_0 = 2000 \, \text{g}$,将它浸没于水中,吸水饱和后取出表面,测得其饱和质量 $m_h = 2350 \, \text{g}$;另外用静水天平测得其浸泡于水中的质量 $m_2 = 1150 \, \text{g}$。求该砖的表观密度、体积吸水率、孔隙率、开口孔隙率和闭口孔隙率。

\begin{remark}
    静水天平的测量原理:物体在空气中的重力和在水中产生的荷载的差,为浮力,可以同时测量质量和体积。
\end{remark}

表观密度(实际上对应浙大版的视密度,只算了开口孔): 
$$
\rho_o = \frac{m_0}{m_0 - m_2} = \frac{2000}{2000 - 1150} = 2.35 \, \text{g/cm}^3
$$

体积密度(实际上对应浙大版的表观密度,因为开口和闭口孔都算上了): 
$$
\rho_\omega = \frac{m_0}{m_1 - m_2} = \frac{2000}{2350 - 1150} = 1.67 \, \text{g/cm}^3
$$

孔隙率:
$$
P = \left( 1 - \frac{\rho_\omega}{\rho_t} \right) \times 100\% = \left( 1 - \frac{1.67}{2.60} \right) \times 100\% = 35.8\%
$$

体积吸水率(只算开口孔):
$$
W_v = P = \left( 1 - \frac{\rho_\omega}{\rho_e} \right) \times 100\% = \left( 1 - \frac{1.67}{2.35} \right) \times 100\% = 28.9\%
$$

开口孔隙率:
$$
P_o = W_v = 28.9\%
$$

闭口孔隙率:
$$
P_c = P - P_o = 35.8\% - 28.9\% = 6.9\%
$$
\end{example}

\begin{example}
    称取堆积密度为1480kg/m³的干砂300g,将此砂加入已装有250mL水的500mL容量瓶内,充分摇动排尽气泡,静止24h小时后加水到刻度,称得总重量为876g;将瓶内砂和水倒出洗净后,再向瓶内重新注水到刻度,此时称得总重量为690 g。再将砂敲碎磨细过筛(0.2 mm)烘干后,取样53.81 g,测得其排水体积V排水为19.85 cm³。试计算该砂的空隙率。

    首先可以轻松算出绝对密度:
    \[
    \rho = \frac{m}{V} = \frac{53.81}{19.85} = 2.71\,\mathrm{g}/\mathrm{cm}^3
    \]
    其次,这里(876-690)g是砂比同体积的水重的质量,所以砂的体积就是300-(876-690)g的水的体积: 
    
    因此表观密度就是(这里你可能疑惑为什么开口闭口啥都没说,为啥就是表观密度了?但是就是,我也不知道为啥,查规范查的):
    \[
    \rho_0 = 300 \times \frac{1}{300 + 690 - 876} = 2.63\text{g}/\text{cm}^3
    \]
    因此,孔隙率就是:
    \[
    P = \left(1 - \frac{\rho_0^{\prime}}{\rho}\right) \times 100\% = \left(1 - \frac{2.63}{2.71}\right) \times 100\% = 43.73\%
    \]
\end{example}

\begin{example}
    评价材料耐水性的指标是什么?

    软化系数:$K_R=\frac{f_b}{f_g}$,大于0.85为耐水材料,类似的有渗透系数,描述渗水能力,$K=\frac{Qd}{AtH}$(H是两侧压强的水头差),$S=10H-1$,渗透等级(H —试件开始渗水时的压力(MPa))。
\end{example}

\begin{example}
    缩写解释题,历年卷有考过的几个高频考点:F50 C25 M7.5 Q235 MU20 42.5R什么意思?

    抗冻能力,用符号Fn表示,其中n即为最大冻融循环次数。 例如F25、F50等。类似需要复习的符号有C25(C是混凝土符号,25是指28天标准抗压强度为25MPa),M7.5(砂浆28天标准养护的抗压强度),42.5(水泥28天标准养护的抗压强度,当然还要比较抗折强度,带R的是早强的),Q是钢材的屈服强度,大于235MPa,然后MU 是砖块的抗压等级,大于20MPa。。
\end{example}

\begin{example}
    有没有导热系数比空气还小的材料?

气凝胶材料由纳米级孔道和极低密度结构组成,极大程度抑制了气体和固体的热传导。
\end{example}


\ifx\allfiles\undefined
\end{document}
\fi