\ifx\allfiles\undefined
\documentclass[12pt, a4paper, oneside, UTF8]{ctexbook}
\def\path{../config}
\input{../config/_config}
\begin{document}
% \input{../config/cover}
\else
\fi

\chapter{动态规划}

\begin{definition}
    阶段:

按时间、空间的特征分解成若干顺序联系的阶段。

\end{definition}

\begin{definition}
    状态:

k阶段开始(或结束)时的客观条件,记为$s_k \in S_k$。$S_k$为k阶段状态集合。

\end{definition}
\begin{definition}
    决策(同时依赖于阶段和状态):

依据状态做出的决定,记为$u_k(s_k) \in D_k(s_k)$,$D_k(s_k)$为状态$s_k$的允许决策集合。
\end{definition}
\begin{definition}
    状态的转移方程:
\[ s_{k+1} = T_k(s_k, u_k(s_k)) \]
\end{definition}

\begin{definition}
    策略

各阶段决策依次构成的决策序列。记为
\[ p_{1,n} = \{u_1(s_1), u_2(s_2), \ldots, u_n(s_n)\} \in P \]
P为允许策略集合。
\end{definition}

\begin{remark}
    动态规划要求问题具有无后效性:

给定某阶段的状态\( s_k \),则以后各阶段的状态\( s_l \)(\( l > k \))都只受\( s_k \)的影响,与之前的状态无关。
\end{remark}

\begin{definition}
    子过程与子策略

后部子过程策略,从k阶段开始到终了阶段的决策子序列,记为
$$
p_{k,n}(s_k) = \{u_k(s_k), u_{k+1}(s_{k+1}), \ldots, u_n(s_n)\} \in P_{k,n}(s_k)
$$
\end{definition}

\begin{definition}
    子策略的指标函数

评价沿子策略$p_{k,n}$过程性能优劣的函数,记为$V_{k,n}(s_k, p_{k,n})$。
\end{definition}

为了实现动态规划的递推结构,要求指标函数具有可分离性

$$
V_{k,n}(s_k, p_{k,n}) = \phi_k(s_k, u_k, V_{k+1,n}(s_{k+1}, p_{k+1,n}))
$$

$\phi_k$ 是 $V_{k+1,n}(s_{k+1}, p_{k+1,n})$ 的严格单调函数。

\begin{remark}
    最优化原理:

最优策略的子策略是相应子问题的最优策略(否则由策略的无后效性,这一定不是最优)。
\end{remark}

\begin{theorem}
    最优化定理

策略 $p_{1,n}^*$ 是最优策略的充要条件是,对于所有的 $k$,都有:
$$ V_{1,n}(s_1, p_{1,n}^*) = \underset{p_{1,k-1} \in P_{1,k-1}}{\text{opt}} \left( V_{1,k-1}(s_1, p_{1,k-1}) + \underset{p_{k,n} \in P_{k,n}}{\text{opt}} V_{k,n}(s_k, p_{k,n}) \right) $$
\end{theorem}




\ifx\allfiles\undefined
\end{document}
\fi