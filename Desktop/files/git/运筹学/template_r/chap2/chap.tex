\ifx\allfiles\undefined
\documentclass[12pt, a4paper, oneside, UTF8]{ctexbook}
\def\path{../config}
\usepackage{amsmath}
\usepackage{amsthm}
\usepackage{array}
\usepackage{amssymb}
\usepackage{graphicx}
\usepackage{mathrsfs}
\usepackage{enumitem}
\usepackage{geometry}
\usepackage[colorlinks, linkcolor=black]{hyperref}
\usepackage{stackengine}
\usepackage{yhmath}
\usepackage{extarrows}
% \usepackage{unicode-math}
\usepackage{esint}
\usepackage{multirow}
\usepackage{fancyhdr}
\usepackage[dvipsnames, svgnames]{xcolor}
\usepackage{listings}
\usepackage{float} % Required for the H float option
\definecolor{mygreen}{rgb}{0,0.6,0}
\definecolor{mygray}{rgb}{0.5,0.5,0.5}
\definecolor{mymauve}{rgb}{0.58,0,0.82}
\definecolor{NavyBlue}{RGB}{0,0,128}
\definecolor{Rhodamine}{RGB}{255,0,255}
\definecolor{PineGreen}{RGB}{0,128,0}

\graphicspath{ {figures/},{../figures/}, {config/}, {../config/} }

\linespread{1.6}

\geometry{
    top=25.4mm, 
    bottom=25.4mm, 
    left=20mm, 
    right=20mm, 
    headheight=2.17cm, 
    headsep=4mm, 
    footskip=12mm
}

\setenumerate[1]{itemsep=5pt,partopsep=0pt,parsep=\parskip,topsep=5pt}
\setitemize[1]{itemsep=5pt,partopsep=0pt,parsep=\parskip,topsep=5pt}
\setdescription{itemsep=5pt,partopsep=0pt,parsep=\parskip,topsep=5pt}

\lstset{
    language=Mathematica,
    basicstyle=\tt,
    breaklines=true,
    keywordstyle=\bfseries\color{NavyBlue}, 
    emphstyle=\bfseries\color{Rhodamine},
    commentstyle=\itshape\color{black!50!white}, 
    stringstyle=\bfseries\color{PineGreen!90!black},
    columns=flexible,
    numbers=left,
    numberstyle=\footnotesize,
    frame=tb,
    breakatwhitespace=false,
} 

\lstset{
    language=TeX, % 设置语言为 TeX
    basicstyle=\ttfamily, % 使用等宽字体
    breaklines=true, % 自动换行
    keywordstyle=\bfseries\color{NavyBlue}, % 关键字样式
    emphstyle=\bfseries\color{Rhodamine}, % 强调样式
    commentstyle=\itshape\color{black!50!white}, % 注释样式
    stringstyle=\bfseries\color{PineGreen!90!black}, % 字符串样式
    columns=flexible, % 列的灵活性
    numbers=left, % 行号在左侧
    numberstyle=\footnotesize, % 行号字体大小
    frame=tb, % 顶部和底部边框
    breakatwhitespace=false % 不在空白处断行
}

% \begin{lstlisting}[language=TeX] ... \end{lstlisting}

% 定理环境设置
\usepackage[strict]{changepage} 
\usepackage{framed}

\definecolor{greenshade}{rgb}{0.90,1,0.92}
\definecolor{redshade}{rgb}{1.00,0.88,0.88}
\definecolor{brownshade}{rgb}{0.99,0.95,0.9}
\definecolor{lilacshade}{rgb}{0.95,0.93,0.98}
\definecolor{orangeshade}{rgb}{1.00,0.88,0.82}
\definecolor{lightblueshade}{rgb}{0.8,0.92,1}
\definecolor{purple}{rgb}{0.81,0.85,1}

\theoremstyle{definition}
\newtheorem{myDefn}{\indent Definition}[section]
\newtheorem{myLemma}{\indent Lemma}[section]
\newtheorem{myThm}[myLemma]{\indent Theorem}
\newtheorem{myCorollary}[myLemma]{\indent Corollary}
\newtheorem{myCriterion}[myLemma]{\indent Criterion}
\newtheorem*{myRemark}{\indent Remark}
\newtheorem{myProposition}{\indent Proposition}[section]

\newenvironment{formal}[2][]{%
	\def\FrameCommand{%
		\hspace{1pt}%
		{\color{#1}\vrule width 2pt}%
		{\color{#2}\vrule width 4pt}%
		\colorbox{#2}%
	}%
	\MakeFramed{\advance\hsize-\width\FrameRestore}%
	\noindent\hspace{-4.55pt}%
	\begin{adjustwidth}{}{7pt}\vspace{2pt}\vspace{2pt}}{%
		\vspace{2pt}\end{adjustwidth}\endMakeFramed%
}

\newenvironment{definition}{\vspace{-\baselineskip * 2 / 3}%
	\begin{formal}[Green]{greenshade}\vspace{-\baselineskip * 4 / 5}\begin{myDefn}}
	{\end{myDefn}\end{formal}\vspace{-\baselineskip * 2 / 3}}

\newenvironment{theorem}{\vspace{-\baselineskip * 2 / 3}%
	\begin{formal}[LightSkyBlue]{lightblueshade}\vspace{-\baselineskip * 4 / 5}\begin{myThm}}%
	{\end{myThm}\end{formal}\vspace{-\baselineskip * 2 / 3}}

\newenvironment{lemma}{\vspace{-\baselineskip * 2 / 3}%
	\begin{formal}[Plum]{lilacshade}\vspace{-\baselineskip * 4 / 5}\begin{myLemma}}%
	{\end{myLemma}\end{formal}\vspace{-\baselineskip * 2 / 3}}

\newenvironment{corollary}{\vspace{-\baselineskip * 2 / 3}%
	\begin{formal}[BurlyWood]{brownshade}\vspace{-\baselineskip * 4 / 5}\begin{myCorollary}}%
	{\end{myCorollary}\end{formal}\vspace{-\baselineskip * 2 / 3}}

\newenvironment{criterion}{\vspace{-\baselineskip * 2 / 3}%
	\begin{formal}[DarkOrange]{orangeshade}\vspace{-\baselineskip * 4 / 5}\begin{myCriterion}}%
	{\end{myCriterion}\end{formal}\vspace{-\baselineskip * 2 / 3}}
	

\newenvironment{remark}{\vspace{-\baselineskip * 2 / 3}%
	\begin{formal}[LightCoral]{redshade}\vspace{-\baselineskip * 4 / 5}\begin{myRemark}}%
	{\end{myRemark}\end{formal}\vspace{-\baselineskip * 2 / 3}}

\newenvironment{proposition}{\vspace{-\baselineskip * 2 / 3}%
	\begin{formal}[RoyalPurple]{purple}\vspace{-\baselineskip * 4 / 5}\begin{myProposition}}%
	{\end{myProposition}\end{formal}\vspace{-\baselineskip * 2 / 3}}


\newtheorem{example}{\indent \color{SeaGreen}{Example}}[section]
\renewcommand{\proofname}{\indent\textbf{\textcolor{TealBlue}{Proof}}}
\newenvironment{solution}{\begin{proof}[\indent\textbf{\textcolor{TealBlue}{Solution}}]}{\end{proof}}

% 自定义命令的文件

\def\d{\mathrm{d}}
\def\R{\mathbb{R}}
%\newcommand{\bs}[1]{\boldsymbol{#1}}
%\newcommand{\ora}[1]{\overrightarrow{#1}}
\newcommand{\myspace}[1]{\par\vspace{#1\baselineskip}}
\newcommand{\xrowht}[2][0]{\addstackgap[.5\dimexpr#2\relax]{\vphantom{#1}}}
\newenvironment{mycases}[1][1]{\linespread{#1} \selectfont \begin{cases}}{\end{cases}}
\newenvironment{myvmatrix}[1][1]{\linespread{#1} \selectfont \begin{vmatrix}}{\end{vmatrix}}
\newcommand{\tabincell}[2]{\begin{tabular}{@{}#1@{}}#2\end{tabular}}
\newcommand{\pll}{\kern 0.56em/\kern -0.8em /\kern 0.56em}
\newcommand{\dive}[1][F]{\mathrm{div}\;\boldsymbol{#1}}
\newcommand{\rotn}[1][A]{\mathrm{rot}\;\boldsymbol{#1}}

% 修改参数改变封面样式,0 默认原始封面、内置其他1、2、3种封面样式
\def\myIndex{0}


\ifnum\myIndex>0
    \input{\path/cover_package_\myIndex}
\fi

\def\myTitle{标题:一份LaTeX笔记模板}
\def\myAuthor{作者名称}
\def\myDateCover{封面日期: \today}
\def\myDateForeword{前言页显示日期: \today}
\def\myForeword{前言标题}
\def\myForewordText{
    
    这是一个基于\LaTeX{}的模板,用于撰写学习笔记。

    模板旨在提供一个简单、易用的框架,以便你能够专注于内容,而不是排版细节,如不是专业者,不建议使用者在模板细节上花费太多时间,而是直接使用模板进行笔记撰写。遇到问题,再进行调整解决。
}
\def\mySubheading{副标题}


\begin{document}
% \input{\path/cover_text_\myIndex.tex}

\newpage
\thispagestyle{empty}
\begin{center}
    \Huge\textbf{\myForeword}
\end{center}
\myForewordText
\begin{flushright}
    \begin{tabular}{c}
        \myDateForeword
    \end{tabular}
\end{flushright}

\newpage
\pagestyle{plain}
\setcounter{page}{1}
\pagenumbering{Roman}
\tableofcontents

\newpage
\pagenumbering{arabic}
\setcounter{chapter}{-1}
\setcounter{page}{1}

\pagestyle{fancy}
\fancyfoot[C]{\thepage}
\renewcommand{\headrulewidth}{0.4pt}
\renewcommand{\footrulewidth}{0pt}








\else
\fi

\chapter{运输问题、目标规划、整数规划}

\section{运输问题}

\begin{definition}
        \textbf{生产平衡问题的一般模型}
        \[
        \text{min } z = \sum_{i=1}^{m} \sum_{j=1}^{n} c_{ij} x_{ij}
        \]
        \[
        \text{s.t. } \sum_{j=1}^{n} x_{ij} = a_i, \quad i=1,2,...,m
        \]
        \[
        \sum_{i=1}^{m} x_{ij} = b_j, \quad j=1,2,...,n
        \]
        \[
        x_{ij} \geq 0, \quad i=1,2,...,m; \quad j=1,2,...,n
        \]
        \[
        \sum_{i=1}^{m} a_i = \sum_{j=1}^{n} b_j
        \]
\end{definition}

\textbf{rank(A) = m+n-1 (约束有1个冗余)的解释:}

m个产地,n个销地,只要满足m+n-1,最后一个自动满足

\begin{enumerate}
    \item 西北角法:优先选择西北角的运输方案
    \item 最小元素法:优先选择运费最小的方案
    \item Vogel法:优先考虑次小运费和最小运费差额大的方案。
\end{enumerate}

前两个省略,伏格尔法的基本思想是,如果次小运费和最小运费的差额很大,\textbf{不早点占住小运费的地方,就会导致差可行解。}

何老师的两个问题:
1、闭回路是否一定存在?
2、闭回路是否唯一?

对于第一个问题,闭回路是肯定存在的,闭回路实际上是单纯形法出基入基在运输问题的特殊体现,单纯形法显然总能出基入基,所以一定可以找到闭回路。

\newpage

对于第二个问题,闭回路是唯一的。

\begin{proof}
假设存在两个环路,对于同一个非基变量,那么可以合并为一个只含基变量的环路

等价于一个基变量对应的方案可以由其他基变量对应的方案表出,这和基变量的定义矛盾。
\end{proof}

\begin{remark}
    除了用闭回路来求检验数和出基入基,实际上还可以用位势法来求检验数,而且快的多,还是解析解法。
\end{remark}

\textbf{位势法求检验数:}
\[
\sigma_j = c_j - C_B B^{-1} P_j = c_j - Y^T P_j
\]
\[
\sigma_{ij} = c_{ij} - [u_1, ..., u_m, v_1, ..., v_n] P_{ij} = c_{ij} - (u_i + v_j)
\]
对应m+n-1个基变量,有 \(\sigma_{ij} = 0\),则:
\[ u_i + v_j = c_{ij} \]
共有m+n个变量,m+n-1个等式,故解不唯一,称为位势。根据位势求非基变量对应的 \(\sigma_{ij}\)。

这里我们常常令$u_1=0$

\section{目标规划}

\begin{definition}
    目标规划的思想和方法

思想:
将定量技术和定性技术结合,
承认矛盾、冲突的合理性,
强调通过协调,达到总体和谐

方法:
软约束+优先级
\end{definition}

\begin{example}
    电视机厂装配25寸和21寸两种彩电,每台电视机需装备时间1小时,每周装配线计划开动40小时,预计每周25寸彩电销售24台,每台可获利80元,每周21寸彩电销售30台,每台可获利40元。
    
    该厂目标:
    
    1. 避免开工不足。
    2. 允许装配线加班,但尽量不超过10小时。
    3. 尽量满足市场需求,尤其是25寸彩电。
    
    解:设 \(x_1, x_2\) 分别表示25寸,21寸彩电产量,目标函数为:
    \[
    \min Z = P_1 d_1^- + P_2 d_2^+ + P_3 (W_{33}^- d_3^- + W_{34}^- d_4^-)
    \]
    其中,\(P_1, P_2, P_3\) 是目标函数的权重系数。
    
    约束条件为:
    \[
    \begin{aligned}
    x_1 + x_2 + d_1^- - d_1^+ &= 40 \quad \text{(上班时间约束)} \\
    x_1 + x_2 + d_2^- - d_2^+ &= 50 \quad \text{(加班时间约束)} \\
    x_1 + d_3^- - d_3^+ &= 24 \quad \text{(25寸市场需求)} \\
    x_2 + d_4^- - d_4^+ &= 30 \quad \text{(21寸市场需求)}
    \end{aligned}
    \]
    
    其中,\(x_1, x_2, d_i^-, d_i^+ \geq 0\),并且有互补松弛条件:
    \[
    d_i^- \cdot d_i^+ = 0 \quad (i = 1, 2, 3, 4)
    \]
\end{example}

\begin{remark}
    优先因子:$P$表明第i个目标的重要程度

    权重系数:$W$表明相同优先因子目标的权重
\end{remark}

何老师的两个问题:

1. 高一级的目标没有满足?低一级的目标是否还有满足机会?

2. \( d_i^- \cdot d_i^+ = 0 \) 为什么可不考虑?

对于第一个,高一级满足了才能满足低一级,这是优先级不可逆性。

对于第二个,从直接上讲,生产不可能既过剩又短缺,所以不需要多此一举。同时,从数学上讲,对于相同的$X$,如果其一不为0,说明没优化到最优解,还可以再优化。

何老师的第三个问题:对于目标规划问题,整体求解和序贯求解哪种更好?

从优先级的不可逆性上讲,二者是等价的。

\section{整数规划问题}

\begin{definition}
    分支定界法

    用于求解整数规划问题,思想是对(极大化的)整数规划问题进行线性松弛,求得最大值,然后在分支的整数规划问题的可行解里求得最小值,直到最后收敛。
\end{definition}

\begin{enumerate}
    \item \textbf{求解整数规划问题}:
      \begin{itemize}
        \item 对于整数规划问题 \( A \),首先解其相应的线性规划问题 \( B \)。
        \item \textbf{若} \( B \) 没有可行解,\textbf{则} \( A \) 也没有可行解,停止。
        \item \textbf{若} \( B \) 有最优解且符合整数条件,\textbf{则} \( B \) 此时的最优解即为 \( A \) 的最优解。
        \item \textbf{若} \( B \) 有最优解但不符合整数条件,\textbf{则} 记此时 \( B \) 的目标函数值为 \( Z \)(上界),那么 \( A \) 的目标函数值 \( z^* \leq Z \)。
      \end{itemize}
  
    \item \textbf{寻找整数可行解}:
      \begin{itemize}
        \item 使用观察法或试探法找到一个整数可行解,通常取简单的组合数进行试探,如:\((0,1), (0,2)\) 等。
        \item 计算该解的目标函数值,并记其为 \( Z \),那么此时有 \( Z \leq z^* \leq Z \)。
      \end{itemize}
  
    \item \textbf{迭代过程}:
      \begin{itemize}
        \item 使用分支定界法的迭代过程进行求解,不断对问题进行分支,选择潜在的解空间进行进一步求解。
        \item 每次通过分支操作将问题分解为子问题
        
        生成新的约束并更新当前的上界和下界。
        \item 在每一分支时,若子问题的下界大于当前的上界,则可舍弃该分支,继续探索其他分支。
        \item 最终通过迭代找到整数解 \( z^* \)。
      \end{itemize}
  \end{enumerate}

\begin{definition}
    割平面法

    割平面的思想是构造可行割,让原来整数规划问题的整数最优解一定满足,但是切割掉了原来可行域的平面,缩小了搜索范围。
\end{definition}

\begin{example}
    求解
\[
\max \quad z = x_1 + x_2
\]
\[
\text{s.t.} \begin{cases}
- x_1 + x_2 \leq 1 \\
3x_1 + x_2 \leq 4 \\
x_1, x_2 \geq 0, \text{整数}
\end{cases}
\]
不考虑整数条件

增加松弛变量
\[
\max \quad z = x_1 + x_2 + 0x_3 + 0x_4
\]
\[
\text{s.t.} \begin{cases}
- x_1 + x_2 + x_3 = 1 \\
3x_1 + x_2 + x_4 = 4 \\
x_1, x_2, x_3, x_4 \geq 0
\end{cases}
\]
\end{example}

这时对于这个线性规划问题,我们可以进行单纯形法的求解,得到变换后的$A$。

(为什么不一开始就用添加可行割的方法?一开始你也不知道是哪个变量不是整数)

\begin{align*}
    x_1 - \frac{1}{4}x_3 + \frac{1}{4}x_4 &= -\frac{3}{4} \\
    x_2 + \frac{3}{4}x_3 + \frac{1}{4}x_4 &= -\frac{7}{4}
    \end{align*}
    
    将系数和常数项均分解成整数与非负真分数之和移项
    
\begin{align*}
    x_1 - x_3 &= \frac{3}{4} - \left( \frac{3}{4}x_3 + \frac{1}{4}x_4 \right) \\
    x_2 - 1 &= \frac{3}{4} - \left( \frac{3}{4}x_3 + \frac{1}{4}x_4 \right)
\end{align*}

\begin{remark}
    左边起到整数约束的作用
\end{remark}

如果你检测下就会发现,添加的可行割让所有整数解都满足了,但是约束到了原来的线性规划的解。

\begin{example}
    如果m个互相排斥的约束条件(≤型):

\[ a_{i1}x_1 + a_{i2}x_2 + \cdots + a_{in}x_n \leq b_i, \text{其中} i = 1, 2, \cdots, m \]

为了保证这m个约束条件只有一个起作用,引入m个0-1变量\( y_i (i = 1, 2, \cdots, m) \)和一个充分大的数M,从而约束条件可以变为:

\[
\begin{cases}
y_1 + y_2 + \cdots + y_m = m - 1 \\
a_{i1}x_1 + a_{i2}x_2 + \cdots + a_{in}x_n \leq b_i + y_iM, \text{其中} i = 1, 2, \cdots, m
\end{cases}
\]

其中,M这个充分大的数保证了当\( y_i = 1 \)时,对应的约束条件是多余的。
\end{example}

\begin{remark}
    0-1型整数规划

    这类问题我们一般使用隐枚举法(可以大致理解为特殊的分支定界法)
\end{remark}

何老师的问题:隐枚举法这样操作是否还能提速?

可以的兄弟!只要调换枚举顺序就可以了。先枚举$X=[0,0,\cdots,0]^T$,再从价格系数高的开始枚举就可以了。

\begin{definition}
    指派问题标准形式

$$
\min z = \sum_{i=1}^{n} \sum_{j=1}^{n} c_{ij} x_{ij}
$$

$$
\text{s.t.} \quad \sum_{j=1}^{n} x_{ij} = 1 \quad \text{每个人有且只有一项工作}
$$

$$
\sum_{i=1}^{n} x_{ij} = 1 \quad \text{每项工作有且只有一个人}
$$

$$
x_{ij} = 0 \text{或} 1 \quad i, j = 1, 2, \ldots, n;
$$

指派问题是特殊的运输问题
\end{definition}

对于指派问题,我们使用匈牙利算法来求解。基本思想是,每一行对应一个产地,每一列对应一个销地,先对每一行进行减法操作,然后对每一列进行减法操作,最后得到一个矩阵。

这个新的成本矩阵和原矩阵最优解一样(每条路都打折,等于没打折)

\begin{definition}
    独立零元素

位于不同行、不同列的零元素,称为独立零元素。

系数矩阵C中独立零元素的个数最多等于能覆盖所有零元素的最少直线数。
\end{definition}

我们可以用独立零元素(实际上只要有一个基准值就行了,不用是零),如果独立零元素个数等于n,说明我们找到了一个最优解。

\begin{remark}
    可能有多个最优解
\end{remark}

\ifx\allfiles\undefined
\end{document}
\fi