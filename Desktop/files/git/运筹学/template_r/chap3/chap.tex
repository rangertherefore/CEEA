\ifx\allfiles\undefined
\documentclass[12pt, a4paper, oneside, UTF8]{ctexbook}
\def\path{../config}
\usepackage{amsmath}
\usepackage{amsthm}
\usepackage{array}
\usepackage{amssymb}
\usepackage{graphicx}
\usepackage{mathrsfs}
\usepackage{enumitem}
\usepackage{geometry}
\usepackage[colorlinks, linkcolor=black]{hyperref}
\usepackage{stackengine}
\usepackage{yhmath}
\usepackage{extarrows}
% \usepackage{unicode-math}
\usepackage{esint}
\usepackage{multirow}
\usepackage{fancyhdr}
\usepackage[dvipsnames, svgnames]{xcolor}
\usepackage{listings}
\usepackage{float} % Required for the H float option
\definecolor{mygreen}{rgb}{0,0.6,0}
\definecolor{mygray}{rgb}{0.5,0.5,0.5}
\definecolor{mymauve}{rgb}{0.58,0,0.82}
\definecolor{NavyBlue}{RGB}{0,0,128}
\definecolor{Rhodamine}{RGB}{255,0,255}
\definecolor{PineGreen}{RGB}{0,128,0}

\graphicspath{ {figures/},{../figures/}, {config/}, {../config/} }

\linespread{1.6}

\geometry{
    top=25.4mm, 
    bottom=25.4mm, 
    left=20mm, 
    right=20mm, 
    headheight=2.17cm, 
    headsep=4mm, 
    footskip=12mm
}

\setenumerate[1]{itemsep=5pt,partopsep=0pt,parsep=\parskip,topsep=5pt}
\setitemize[1]{itemsep=5pt,partopsep=0pt,parsep=\parskip,topsep=5pt}
\setdescription{itemsep=5pt,partopsep=0pt,parsep=\parskip,topsep=5pt}

\lstset{
    language=Mathematica,
    basicstyle=\tt,
    breaklines=true,
    keywordstyle=\bfseries\color{NavyBlue}, 
    emphstyle=\bfseries\color{Rhodamine},
    commentstyle=\itshape\color{black!50!white}, 
    stringstyle=\bfseries\color{PineGreen!90!black},
    columns=flexible,
    numbers=left,
    numberstyle=\footnotesize,
    frame=tb,
    breakatwhitespace=false,
} 

\lstset{
    language=TeX, % 设置语言为 TeX
    basicstyle=\ttfamily, % 使用等宽字体
    breaklines=true, % 自动换行
    keywordstyle=\bfseries\color{NavyBlue}, % 关键字样式
    emphstyle=\bfseries\color{Rhodamine}, % 强调样式
    commentstyle=\itshape\color{black!50!white}, % 注释样式
    stringstyle=\bfseries\color{PineGreen!90!black}, % 字符串样式
    columns=flexible, % 列的灵活性
    numbers=left, % 行号在左侧
    numberstyle=\footnotesize, % 行号字体大小
    frame=tb, % 顶部和底部边框
    breakatwhitespace=false % 不在空白处断行
}

% \begin{lstlisting}[language=TeX] ... \end{lstlisting}

% 定理环境设置
\usepackage[strict]{changepage} 
\usepackage{framed}

\definecolor{greenshade}{rgb}{0.90,1,0.92}
\definecolor{redshade}{rgb}{1.00,0.88,0.88}
\definecolor{brownshade}{rgb}{0.99,0.95,0.9}
\definecolor{lilacshade}{rgb}{0.95,0.93,0.98}
\definecolor{orangeshade}{rgb}{1.00,0.88,0.82}
\definecolor{lightblueshade}{rgb}{0.8,0.92,1}
\definecolor{purple}{rgb}{0.81,0.85,1}

\theoremstyle{definition}
\newtheorem{myDefn}{\indent Definition}[section]
\newtheorem{myLemma}{\indent Lemma}[section]
\newtheorem{myThm}[myLemma]{\indent Theorem}
\newtheorem{myCorollary}[myLemma]{\indent Corollary}
\newtheorem{myCriterion}[myLemma]{\indent Criterion}
\newtheorem*{myRemark}{\indent Remark}
\newtheorem{myProposition}{\indent Proposition}[section]

\newenvironment{formal}[2][]{%
	\def\FrameCommand{%
		\hspace{1pt}%
		{\color{#1}\vrule width 2pt}%
		{\color{#2}\vrule width 4pt}%
		\colorbox{#2}%
	}%
	\MakeFramed{\advance\hsize-\width\FrameRestore}%
	\noindent\hspace{-4.55pt}%
	\begin{adjustwidth}{}{7pt}\vspace{2pt}\vspace{2pt}}{%
		\vspace{2pt}\end{adjustwidth}\endMakeFramed%
}

\newenvironment{definition}{\vspace{-\baselineskip * 2 / 3}%
	\begin{formal}[Green]{greenshade}\vspace{-\baselineskip * 4 / 5}\begin{myDefn}}
	{\end{myDefn}\end{formal}\vspace{-\baselineskip * 2 / 3}}

\newenvironment{theorem}{\vspace{-\baselineskip * 2 / 3}%
	\begin{formal}[LightSkyBlue]{lightblueshade}\vspace{-\baselineskip * 4 / 5}\begin{myThm}}%
	{\end{myThm}\end{formal}\vspace{-\baselineskip * 2 / 3}}

\newenvironment{lemma}{\vspace{-\baselineskip * 2 / 3}%
	\begin{formal}[Plum]{lilacshade}\vspace{-\baselineskip * 4 / 5}\begin{myLemma}}%
	{\end{myLemma}\end{formal}\vspace{-\baselineskip * 2 / 3}}

\newenvironment{corollary}{\vspace{-\baselineskip * 2 / 3}%
	\begin{formal}[BurlyWood]{brownshade}\vspace{-\baselineskip * 4 / 5}\begin{myCorollary}}%
	{\end{myCorollary}\end{formal}\vspace{-\baselineskip * 2 / 3}}

\newenvironment{criterion}{\vspace{-\baselineskip * 2 / 3}%
	\begin{formal}[DarkOrange]{orangeshade}\vspace{-\baselineskip * 4 / 5}\begin{myCriterion}}%
	{\end{myCriterion}\end{formal}\vspace{-\baselineskip * 2 / 3}}
	

\newenvironment{remark}{\vspace{-\baselineskip * 2 / 3}%
	\begin{formal}[LightCoral]{redshade}\vspace{-\baselineskip * 4 / 5}\begin{myRemark}}%
	{\end{myRemark}\end{formal}\vspace{-\baselineskip * 2 / 3}}

\newenvironment{proposition}{\vspace{-\baselineskip * 2 / 3}%
	\begin{formal}[RoyalPurple]{purple}\vspace{-\baselineskip * 4 / 5}\begin{myProposition}}%
	{\end{myProposition}\end{formal}\vspace{-\baselineskip * 2 / 3}}


\newtheorem{example}{\indent \color{SeaGreen}{Example}}[section]
\renewcommand{\proofname}{\indent\textbf{\textcolor{TealBlue}{Proof}}}
\newenvironment{solution}{\begin{proof}[\indent\textbf{\textcolor{TealBlue}{Solution}}]}{\end{proof}}

% 自定义命令的文件

\def\d{\mathrm{d}}
\def\R{\mathbb{R}}
%\newcommand{\bs}[1]{\boldsymbol{#1}}
%\newcommand{\ora}[1]{\overrightarrow{#1}}
\newcommand{\myspace}[1]{\par\vspace{#1\baselineskip}}
\newcommand{\xrowht}[2][0]{\addstackgap[.5\dimexpr#2\relax]{\vphantom{#1}}}
\newenvironment{mycases}[1][1]{\linespread{#1} \selectfont \begin{cases}}{\end{cases}}
\newenvironment{myvmatrix}[1][1]{\linespread{#1} \selectfont \begin{vmatrix}}{\end{vmatrix}}
\newcommand{\tabincell}[2]{\begin{tabular}{@{}#1@{}}#2\end{tabular}}
\newcommand{\pll}{\kern 0.56em/\kern -0.8em /\kern 0.56em}
\newcommand{\dive}[1][F]{\mathrm{div}\;\boldsymbol{#1}}
\newcommand{\rotn}[1][A]{\mathrm{rot}\;\boldsymbol{#1}}

% 修改参数改变封面样式,0 默认原始封面、内置其他1、2、3种封面样式
\def\myIndex{0}


\ifnum\myIndex>0
    \input{\path/cover_package_\myIndex}
\fi

\def\myTitle{标题:一份LaTeX笔记模板}
\def\myAuthor{作者名称}
\def\myDateCover{封面日期: \today}
\def\myDateForeword{前言页显示日期: \today}
\def\myForeword{前言标题}
\def\myForewordText{
    
    这是一个基于\LaTeX{}的模板,用于撰写学习笔记。

    模板旨在提供一个简单、易用的框架,以便你能够专注于内容,而不是排版细节,如不是专业者,不建议使用者在模板细节上花费太多时间,而是直接使用模板进行笔记撰写。遇到问题,再进行调整解决。
}
\def\mySubheading{副标题}


\begin{document}
% \input{\path/cover_text_\myIndex.tex}

\newpage
\thispagestyle{empty}
\begin{center}
    \Huge\textbf{\myForeword}
\end{center}
\myForewordText
\begin{flushright}
    \begin{tabular}{c}
        \myDateForeword
    \end{tabular}
\end{flushright}

\newpage
\pagestyle{plain}
\setcounter{page}{1}
\pagenumbering{Roman}
\tableofcontents

\newpage
\pagenumbering{arabic}
\setcounter{chapter}{-1}
\setcounter{page}{1}

\pagestyle{fancy}
\fancyfoot[C]{\thepage}
\renewcommand{\headrulewidth}{0.4pt}
\renewcommand{\footrulewidth}{0pt}








\else
\fi

\chapter{非线性凸规划}

\begin{definition}
    非线性规划的数学模型

$$
\min f(X)
$$
$$
\text{s.t.} \quad h_i(X) = 0 \quad i = 1, 2, \ldots, m
$$
$$
g_j(X) \geq 0 \quad j = 1, 2, \ldots, l
$$
$$
X \in R^n
$$
\end{definition}

何老师的问题:非线性规划找的是局部最小点还是全局最小点?

实际上对于非线性凸规划,我们找到的局部最小点就是全局最小点。
对于复杂的非线性规划,实际上我们不确定找到的是局部最小点还是全局最小点。

\begin{definition}
    凸函数与凹函数

凸函数为满足下列条件的函数:
  $$
  f[\lambda X_1 + (1-\lambda)X_2] \leq \lambda f(X_1) + (1-\lambda)f(X_2) \quad 0 < \lambda < 1 \quad X_1, X_2 \in R^n
  $$
\end{definition}

\begin{proof}
    证明凸函数的局部最小点就是全局最小点

    对于任意 \( \theta \in [0, 1] \),验证:
\[
f(\theta x + (1 - \theta)y) \leq \theta f(x) + (1 - \theta) f(y)
\]

展开左边并整理后,比较两边的差值为:
\[
\frac{\theta(1-\theta)}{2}(x - y)^T A (x - y) \leq 0
\]

由于 \( A \) 半正定且 \( \theta(1-\theta) \geq 0 \),不等式成立,故 \( f \) 是凸函数。
\end{proof}

\textbf{凸函数的性质}

\begin{enumerate}
    \item 凸函数非负线性组合仍为凸函数
      \[
      \gamma_1 f_1(X) + \gamma_2 f_2(X) \quad \gamma_i \geq 0
      \]
    
    \item 若 \( f(X) \) 是定义在凸集 \( R_C \) 上的凸函数,则其 \( \beta \) 水平集 \( S_{\beta} \) 为凸集。
      \[
      S_{\beta} = \{ X \mid f(X) \leq \beta, X \in R_C \} \quad \beta \text{水平集}
      \]
    
    \item 对于凸函数 \( f(X) \),若存在 \( X^* \in R^n \) 满足
      \[
      \nabla f(X^*)^T (X - X^*) \geq 0 \quad \forall X \in R^n
      \]
      则 \( X^* \) 为 \( f(X) \) 的全局最小点。充要条件!
\end{enumerate}

(1)(3)显然,(3)先证明是局部最小点,然后证明的全局最小点。

对于(2),直接使用定义是可以证明的。

\begin{definition}
    凸规划问题

$$
\min f(X)
$$
s.t. $h_i(X) = 0 \quad i = 1, 2, ..., m \quad \text{线性函数}$
$$
g_j(X) \geq 0 \quad j = 1, 2, ..., l \quad \text{凹函数}
$$
$$
X \in R^n
$$
\end{definition}

实际上第二个凹函数的要求,如果是左右同乘一个负号,实际上是一个凸函数。
所以这个规划问题的可行域也是个凸集(线性函数是凸函数也是凹函数)。

\begin{theorem}
    若目标函数为严格凸函数,则如果全局最优解存在必为唯一全局最优解。(反证法)

    如果不是严格,我们用定义可以说明凸函数的最优解解集一定是各个最优解的凸包
\end{theorem}

\begin{definition}
    海森矩阵
    \[
    \nabla^2 f(X) = 
\begin{bmatrix}
\frac{\partial^2 f(X)}{\partial x_1^2} & \frac{\partial^2 f(X)}{\partial x_1 \partial x_2} & \cdots & \frac{\partial^2 f(X)}{\partial x_1 \partial x_n} \\
\frac{\partial^2 f(X)}{\partial x_2 \partial x_1} & \frac{\partial^2 f(X)}{\partial x_2^2} & \cdots & \frac{\partial^2 f(X)}{\partial x_2 \partial x_n} \\
\vdots & \vdots & \ddots & \vdots \\
\frac{\partial^2 f(X)}{\partial x_n \partial x_1} & \frac{\partial^2 f(X)}{\partial x_n \partial x_2} & \cdots & \frac{\partial^2 f(X)}{\partial x_n^2}
\end{bmatrix}
\]
\end{definition}

\[
f(X + \Delta X) = f(X) + \nabla f(X)^T \Delta X + \frac{1}{2} \Delta X^T \nabla^2 f(X) \Delta X + O(||\Delta X||^2)
\]

如果某处梯度为0,海森矩阵为半正定,实际上这只是局部极小值的必要条件。除非加强为领域梯度均大于等于0。

\begin{definition}
    拉格朗日乘数法(针对等式约束):

定义Lagrange函数
$$
\min L(X, \lambda) = f(X) + \lambda^T h(X) = f(X) + \sum_{i=1}^{m} \lambda_i h_i(X)
$$
$$
\lambda = \begin{bmatrix} \lambda_1 & \lambda_2 & \cdots & \lambda_m \end{bmatrix}^T
$$

求无约束极值问题:

$$
\left. \frac{\partial L(X, \lambda)}{\partial X} \right|_{X^*} = 0 \quad \Rightarrow \quad \nabla f(X^*) + \sum_{i=1}^{m} \lambda_i \nabla h_i(X^*) = 0
$$
$$
\left. \frac{\partial L(X, \lambda)}{\partial \lambda} \right|_{\lambda^*} = 0 \quad \Rightarrow \quad h_i(X^*) = 0 \quad i = 1, 2, \ldots, m
$$
\end{definition}

\textbf{不等式约束极值问题:}
\[
\min f(X)
\]
\[
\text{s.t.} \quad g_j(X) \geq 0 \quad j = 1, 2, \ldots, l
\]
\[
X \in R^n
\]

\begin{definition}
    定义:令目标函数值 \( f(X^{(0)}) \) 下降的方向,满足
\[
\exists \lambda_0 > 0, \text{ 对于 } \forall \lambda \in (0, \lambda_0], \text{ 有 } f(X^{(0)} + \lambda P) < f(X^{(0)})
\]
判别条件: \(\nabla f(X^{(0)})^T P < 0\)
\end{definition}

\begin{theorem}
Gordan 引理

\( A_j \) 为一组已知向量,不存在向量 \( P \),使
\[
A_j^T P < 0 \quad j = 1, 2, ..., l
\]

同时成立的充要条件:

存在不全为零的非负实数 \( \mu_j \geq 0 \),使
\[
\sum_{j=1}^{l} \mu_j A_j = 0
\]

几何含义:\( A_j \) 不可能分布在任何超平面的同一侧。
\end{theorem}

\begin{proof}
    用Farkas引理证明Gordan引理

    Farkas引理:对于矩阵$A \in \mathbb{R}^{m \times n}$和向量$b \in \mathbb{R}^m$,以下两个系统恰有一个有解:

1. 存在$x \geq 0$使得$Ax = b$;

2. 存在$y \in \mathbb{R}^m$使得$y^\top A \geq 0^\top$且$y^\top b < 0$。

\textbf{证明Gordon引理}:

\begin{enumerate}
    \item \textbf{构造辅助系统}:
    
    将系统A加强为存在满足归一化条件的解。令 \( e = (1,1,\dots,1)^\top \),定义新系统:
    \[
    \exists x \geq 0,\quad 
    \begin{cases}
        Ax = 0 \\
        e^\top x = 1
    \end{cases}
    \]
    若此系统有解,则原系统A有非零解;反之则无。

    \item \textbf{应用Farkas引理}:
    
    将上述系统写作矩阵形式 \( \begin{bmatrix} A \\ e^\top \end{bmatrix} x = \begin{bmatrix} 0 \\ 1 \end{bmatrix} \)。根据Farkas引理,恰有以下之一成立:
    \begin{enumerate}
        \item 存在 \( x \geq 0 \) 满足方程
        \item 存在 \( \begin{bmatrix} y \\ \lambda \end{bmatrix} \in \mathbb{R}^{m+1} \) 使得:
        \[
        y^\top A + \lambda e^\top \geq 0^\top \quad \text{且} \quad y^\top 0 + \lambda \cdot 1 < 0
        \]
    \end{enumerate}
    第二个不等式简化为 \( \lambda < 0 \)。

    \item \textbf{分析对偶条件}:
    
    若原系统无解,则存在 \( y \) 和 \( \lambda < 0 \) 使得:
    \[
    y^\top A_j + \lambda \geq 0 \quad (\forall j=1,\dots,n)
    \]
    令 \( \mu = -\lambda > 0 \),则有:
    \[
    y^\top A_j \geq \mu > 0 \quad (\forall j)
    \]
    即 \( A^\top y > 0 \),故系统B成立。

    \item \textbf{反方向证明}:
    
    若系统B有解(存在 \( y \) 使 \( A^\top y > 0 \)),假设系统A也有解(存在非零 \( x \geq 0 \) 使 \( Ax = 0 \)),则:
    \[
    y^\top A x = (A^\top y)^\top x > 0 \quad (\text{因 } A^\top y > 0 \text{ 且 } x \geq 0, \neq 0)
    \]
    但左边等于 \( y^\top 0 = 0 \),矛盾。故系统B存在时系统A无解。
\end{enumerate}

综上,系统A与系统B恰有一个成立,Gordon引理得证。
\end{proof}

\begin{theorem}
    Fritz John条件

若X*是局部极小点,存在不全为零的非负实数 \( \mu_j \geq 0, j \in J(X*) \cup \{0\} \),使
\[
\mu_0 \nabla f(X*) - \sum_{j \in J} \mu_j \nabla g_j(X*) = 0
\]
\end{theorem}

\begin{proof}
    证明:Gordan引理的直接推论(不存在一个可行的下降方向)
\[
\nabla f(X*)^T P < 0
\]
\[
-\nabla g_j(X*)^T P < 0
\]
\end{proof}

\begin{remark}
    等价Fritz John条件(将$J$显式化)

若$X^*$是局部极小点,则存在不全为零的非负实数 $\mu_j \geq 0$,$j = 0, 1, \ldots, l$,使下列条件成立:
$$
\begin{cases}
\mu_0 \nabla f(X^*) - \sum_{j=1}^{l} \mu_j \nabla g_j(X^*) = 0 & \text{Lagrange驻点条件} \\
\mu_j g_j(X^*) = 0 & j = 1, 2, \ldots, l \quad \text{互补松弛条件} \\
\mu_j \geq 0 & \sum_{j=0}^{m} \mu_j \neq 0 \quad j = 0, 1, \ldots, l \quad \text{强非负条件}
\end{cases}
$$
上述条件隐含了如下事实:$\mu_j = 0 \quad j \notin J$
\end{remark}

\begin{definition}
    Kuhn-Tucker条件

若Fritz John条件的$\mu_0\ge0$,则可推出KKT条件的数学形式:
$$
\begin{cases}
\nabla f(X^*) - \sum_{j=1}^{l} \mu_j^* \nabla g_j(X^*) = 0 & \text{Lagrange驻点条件} \\
\mu_j^* g_j(X^*) = 0 & j = 1, 2, \ldots, l \quad \text{互补松弛条件} \\
\mu_j^* \geq 0 & j = 1, 2, \ldots, l \quad \text{非负条件}
\end{cases}
$$
Karush(1939), Kuhn-Tucker (1951)
\end{definition}

何老师的问题:什么条件下$\mu_0\ge0$?

在加上约束规格限制时$\mu_0\ge0$,比如说正则条件

\begin{itemize}
    \item \textbf{Fritz John点}:满足Fritz John条件(允许目标函数乘子为零)
    \item \textbf{KKT点}:满足KKT条件(需约束规格,乘子非负)
    \item \textbf{正则KKT点}:满足LICQ或严格互补的KKT点
    \item \textbf{极值点}:局部最优解
    
    \vspace{1em}
    \begin{itemize}
        \item \textbf{包含关系}:
        \[
        \text{Fritz John点} \supsetneq \text{KKT点} \supsetneq \text{正则KKT点}
        \]
        \item \textbf{极值点归属}:
        \begin{itemize}
            \item 若满足约束规格:$\text{极值点} \subset \text{KKT点}$
            \item 若不满足约束规格:$\text{极值点} \subset \text{Fritz John点}$
        \end{itemize}
        \item \textbf{非极值点存在性}:
        \begin{itemize}
            \item 存在非极值的KKT点(如鞍点)
            \item 存在非极值的Fritz John点(目标乘子为零)
        \end{itemize}
    \end{itemize}
\end{itemize}

等价不等式约束极值问题(等式约束改写成不等式)
\[
\min f(X)
\]
\[
\text{s.t.} \quad h_i(X) \geq 0 \quad i = 1, 2, \ldots, m
\]
\[
-h_i(X) \geq 0 \quad i = 1, 2, \ldots, m
\]
\[
g_j(X) \geq 0 \quad j = 1, 2, \ldots, l
\]
\[
X \in R^n
\]
问题:此时约束是否满足正则条件?

显然不满足,等式约束的梯度函数线性相关

\textbf{一般约束的KKT条件的推导}

\begin{equation*}
    \mu_0^* \nabla f(X^*) - \sum_{j=1}^l \mu_j^* \nabla g_j(X^*) - \sum_{i=1}^m \mu_i'^* \nabla h_i(X^*) + \sum_{i=1}^m \mu_i''^* \nabla h_i(X^*) = 0
    \end{equation*}
    
    \begin{equation*}
    \mu_0^* \nabla f(X^*) - \sum_{j=1}^l \mu_j^* \nabla g_j(X^*) - \sum_{i=1}^m (\mu_i'^* - \mu_i''^*) \nabla h_i(X^*) = 0
    \end{equation*}
    
    \begin{equation*}
    \mu_0^* \nabla f(X^*) - \sum_{j=1}^l \mu_j^* \nabla g_j(X^*) - \sum_{i=1}^m \gamma_i^* \nabla h_i(X^*) = 0
    \end{equation*}
    
    \begin{equation*}
    \mu_i'^* \geq 0 \quad \mu_i''^* \geq 0 \quad \Rightarrow \quad \gamma_i^* = \mu_i'^* - \mu_i''^* \text{无符号约束} \quad i = 1, 2, \ldots, p
    \end{equation*}
    
注意 \(\mu_0, \mu_j, \gamma_i\) 不可同时为0!

\begin{definition}
    一般约束下的Fritz John条件

若X*是局部极小点,存在不全为零的μj (j=0,1,2,...,m)和γi (i=0,1,2,...,p),满足

$$
\mu_0^* \nabla f(X^*) - \sum_{j=1}^{l} \mu_j^* \nabla g_j(X^*) - \sum_{i=1}^{m} \gamma_i^* \nabla h_i(X^*) = 0 \quad \text{Lagrange驻点条件}
$$
$$
\mu_j^* g_j(X^*) = 0 \quad j = 1, 2, ..., m \quad \text{互补松弛条件}
$$
$$
\mu_j^* \geq 0 \quad j = 0, 1, ..., m 
$$
$$
\sum_{j=0}^{l} \mu_j^* + \sum_{i=1}^{m} |\gamma_i^*| \neq 0 \quad \text{强非负条件}
$$
$$
\gamma_i^* h_i(X^*) = 0 \quad i = 1, 2, ..., p \quad \text{等式互补松弛条件}
$$
\end{definition}

\begin{definition}
    KKT条件的矩阵形式
$$
\nabla f(X^*) - \nabla h(X^*) \gamma^* - \nabla g(X^*) \mu^* = 0
$$
$$
\mu^* \odot g(X^*) = 0 \quad \Leftrightarrow \quad \mu_j^* g_j(X^*) = 0 \quad j = 1, 2, \ldots, l
$$
其中
$$
g(X^*) \geq 0
$$
$$
h(X^*) = 0
$$
$$
\mu^* \geq 0
$$
\end{definition}

\begin{theorem}
凸规划下的KKT条件为最优解的充要条件
\end{theorem}

\begin{proof}
    KKT条件的充分性证明
$$
\nabla f(X^*) - \sum_{i=1}^{m} \gamma_i^* \nabla h_i(X^*) - \sum_{j=1}^{l} \mu_j^* \nabla g_j(X^*) = 0 \quad \mu_j^* g_j(X^*) = 0 \quad \mu_j^* \geq 0
$$
$f(X)$是凸函数 $\Rightarrow f(X) \geq f(X^*) + \nabla f(X^*)^T (X - X^*)$
$$
= f(X^*) + \sum_{i=1}^{m} \gamma_i^* \nabla h_i(X^*)^T (X - X^*) + \sum_{j=1}^{l} \mu_j^* \nabla g_j(X^*)^T (X - X^*)
$$
$h_i(X)$线性函数且 $h_i(X) = 0$ $\Rightarrow \nabla h_i(X^*)^T (X - X^*) = h_i(X) - h_i(X^*) = 0$
$g_j(X)$是凹函数 $\Rightarrow \nabla g_j(X^*)^T (X - X^*) \geq g_j(X) - g_j(X^*)$
根据互补松弛性条件 $\mu_j^* g_j(X^*) = 0$ 且 $g_j(X) \geq 0$,有
$$
f(X) \geq f(X^*) + \sum_{j=1}^{l} \mu_j^* \left[ g_j(X) - g_j(X^*) \right] = f(X^*) + \sum_{j=1}^{l} \mu_j^* g_j(X) \geq f(X^*)
$$
\end{proof}



\ifx\allfiles\undefined
\end{document}
\fi