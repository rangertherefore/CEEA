\ifx\allfiles\undefined
\documentclass[12pt, a4paper, oneside, UTF8]{ctexbook}
\def\path{../config}
\input{../config/_config}
\begin{document}
\input{../config/cover}
\else
\fi

\chapter{对策论}

\begin{definition}
\label{def:game}
    \textbf{博弈}是指在一个特定的环境中,多个参与者(称为博弈者)通过选择策略来影响结果的过程。博弈论研究这些博弈的性质、策略选择和结果。
    
    博弃模型 \(G = \{I, II; S_1, S_2; A\}\)
\end{definition}

\[
\mathbf{A} = \begin{bmatrix}
a_{11} & a_{12} & \cdots & a_{1n} \\
a_{21} & a_{22} & \cdots & a_{2n} \\
\vdots & \vdots & \ddots & \vdots \\
a_{m1} & a_{m2} & \cdots & a_{mn}
\end{bmatrix}
\]

如果存在

$$
\max_{i} \min_{j} a_{ij} = \min_{j} \max_{i} a_{ij} = a_{i^* j^*} = V_G
$$

则 $(\alpha_{i^*}, \beta_{j^*})$ 为矩阵博弈的最优纯策略对,也称为最优局势。$V_G$ 称为博弈值。

矩形博弈最优纯策略对存在的充要条件是存在鞍点。

$$ a_{ij^*} \leq a_{i^*j^*} \leq a_{i^*j} \quad \text{(鞍点条件)} $$
$$ \max_{i} a_{ij^*} \leq a_{i^*j^*} \leq \min_{j} a_{i^*j} $$

\begin{proof}
    \textbf{必要性}
    $$
    \max_i \min_j a_{ij} = \min_j \max_i a_{ij}
    $$
    $$
    \exists i^*, j^* \quad \min_j a_{i^* j} = \max_i \min_j a_{ij} = \min_j \max_i a_{ij} = \max_i a_{i j^*}
    $$
    $$
    a_{i^* j^*} \geq \min_j a_{i^* j} = \max_i \min_j a_{ij} = \min_j \max_i a_{ij} = \max_i a_{i j^*} \geq a_{i^* j^*}
    $$
    $$
    \min_j a_{i^* j} = \max_i a_{i j^*} = a_{i^* j^*} \quad \Rightarrow \quad a_{i j^*} \leq a_{i^* j^*} \leq a_{i^* j}
    $$
    \textbf{充分性}
    \[ a_{ij*} \leq a_{i*j*} \leq a_{i*j} \]
    \[ \max_{i} a_{ij*} \leq a_{i*j*} \leq \min_{j} a_{i*j} \]
    \[ \min_{j} \max_{i} a_{ij} \leq a_{i*j*} \leq \max_{i} \min_{j} a_{ij} \]
    又根据 \[ \max_{i} \min_{j} a_{ij} \leq \min_{j} \max_{i} a_{ij} \]
    \[ \max_{i} \min_{j} a_{ij} = \min_{j} \max_{i} a_{ij} \]
\end{proof}

当我们找到这样一组决策时,对于任何其他决策和双方决策者,我们都不能得到更好的结果,这就是鞍点的意思。

\begin{remark}
    纳什均衡解 $\leftrightarrow$ 最优纯策略对
\end{remark}

多个均衡解的性质:
\begin{itemize}
    \item 无差别性
    \item 可交换性
    \item 存在无策略解,此时选择随机策略
\end{itemize}

\begin{definition}
    博奕模型 $G^* = \{S_1^*, S_2^*; E\}$

    $\triangleright$ 混合策略集

    $$ S_1^* = \left\{ x \in R^m \mid x_i \geq 0, \ i = 1, 2, \cdots, m; \sum_{i=1}^m x_i = 1 \right\} $$

    $x_i$ 为局中人I执行纯策略 $\alpha_i$ 的概率

    $$ S_2^* = \left\{ y \in R^n \mid y_j \geq 0, \ j = 1, 2, \cdots, n; \sum_{j=1}^n y_j = 1 \right\} $$

    $y_j$ 为局中人II执行纯策略 $\beta_i$ 的概率

    $\triangleright$ 局中人I的赢得函数: $E(x, y) = x^T A y = \sum_{i=1}^m \sum_{j=1}^n a_{ij} x_i y_j$

    $\triangleright$ 局中人II的赢得函数: $-E(x, y)$
\end{definition}

如果双方都是理性决策

- 局中人I的最大预期赢得:
  $$
  \max_{x \in S_1} \min_{y \in S_2} E(x, y)
  $$

- 局中人II的最小预期损失:
  $$
  \min_{y \in S_2} \max_{x \in S_1} E(x, y)
  $$

两者关系:
$$
\max_{x \in S_1} \min_{y \in S_2} E(x, y) \leq \min_{y \in S_2} \max_{x \in S_1} E(x, y)
$$

混合策略
$$
\mathbf{x} = [x_1, x_2, \cdots, x_m]^T \quad \mathbf{y} = [y_1, y_2, \cdots, y_n]^T
$$

\begin{theorem}
    最优混合策略对/均衡解存在的充要条件:存在鞍点
    
    $$E(x, y^*) \leq E(x^*, y^*) \leq E(x^*, y)$$
\end{theorem}

\begin{proof}
    一定存在混合策略对意义下的矩阵博弈均衡解

$\exists$x*,y*满足:
$$\sum_{i=1}^{m}\sum_{j=1}^{n}a_{ij}y_{j}^{*}x_{i}=E(x,y^{*})\leq E(x^{*},y^{*})\leq E(x^{*},y)=\sum_{j=1}^{n}\sum_{i=1}^{m}a_{ij}x_{i}^{*}y_{j}^{*}$$

鞍点条件
$$\sum_{j=1}^{n}a_{ij}y_{j}^{*}\leq E(x^{*},y^{*})\leq \sum_{i=1}^{m}a_{ij}x_{i}^{*} \quad \forall i=1,2,\cdots,m \quad \forall j=1,2,\cdots,n$$

一方采用纯策略,一方采用最优混合策略
$$\exists v=E(x^{*},y^{*})$$ 使下面2组不等式均有解

$$\sum_{i=1}^{m}a_{ij}x_{i}^{*}\geq v \quad j=1,2,\cdots,n$$
$$\sum_{i=1}^{m}x_{i}^{*}=1 \quad x_{i}^{*}\geq 0 \quad i=1,2,\cdots,m$$

$$\sum_{j=1}^{n}a_{ij}y_{j}^{*}\leq v \quad i=1,2,\cdots,m$$
$$\sum_{j=1}^{n}y_{j}^{*}=1 \quad y_{j}^{*}\geq 0 \quad j=1,2,\cdots,n$$

使得对偶问题均有可行解,且 \( w^* = v^* = E(x^*, y^*) \)

可以验证
\[
\mathbf{x} = [1 \quad 0 \quad \cdots \quad 0] \quad w = \min a_{1j}
\]
\[
\mathbf{y} = [1 \quad 0 \quad \cdots \quad 0] \quad v = \max a_{i1}
\]
原问题有可行解同时对偶问题也有可行解

故有
\[
\sum_{j=1}^{n} a_{ij} y_j^* = v^* = w^* = \sum_{i=1}^{m} a_{ij} x_i^* \quad \text{强对偶性}
\]

又根据:
\[
E(\mathbf{x}^*, \mathbf{y}^*) = \sum_{j=1}^{n} \sum_{i=1}^{m} a_{ij} x_i^* y_j^* = \sum_{i=1}^{m} \left( \sum_{j=1}^{n} a_{ij} y_j^* \right) x_i^* \leq v^* \sum_{i=1}^{m} x_i^* = v^*
\]
\[
E(\mathbf{x}^*, \mathbf{y}^*) = \sum_{j=1}^{n} \sum_{i=1}^{m} a_{ij} x_i^* y_j^* = \sum_{j=1}^{n} \left( \sum_{i=1}^{m} a_{ij} x_i^* \right) y_j^* \geq w^* \sum_{j=1}^{n} y_j^* = w^*
\]

\[
w^* = v^* = E(\mathbf{x}^*, \mathbf{y}^*)
\]

\end{proof}

\begin{theorem}
    从两个互为对偶的线性规划问题里,我们有互补松弛性
\[
x_i^* > 0 \implies \sum_{j=1}^{n} a_{ij} y_j^* = v^* = E(x^*, y^*)
\]

\[
y_j^* > 0 \implies \sum_{i=1}^{m} a_{ij} x_i^* = w^* = E(x^*, y^*)
\]

\[
\sum_{j=1}^{n} a_{ij} y_j^* < v^* = E(x^*, y^*) \implies x_i^* = 0
\]

\[
\sum_{i=1}^{m} a_{ij} x_i^* > w^* = E(x^*, y^*) \implies y_j^* = 0
\]

如果某条纯策略在均衡解中有被选择的可能,则对手的最优混合策略在该纯策略下的赢得值不会比 \(V_G\) 更好。

如果某条纯策略下对手的最优混合策略的赢得值比 \(V_G\) 更好,则该纯策略在均衡解中无被选择可能。
\end{theorem}

解集T(G):博弈G的均衡解集合。

赢得矩阵严格单调变换下的解集不变性

博弈:
\[ G_1 = \{S_1, S_2; A_1\} \]
\[ G_2 = \{S_1, S_2; A_2\} \]

\[ A_2 = A_1 + L * 1_{m \times n} \quad \Longrightarrow \quad T(G_1) = T(G_2) \quad V_{G_1} = V_{G_2} + L \]

\[ A_2 = aA_1 \quad a > 0 \quad \Longrightarrow \quad T(G_1) = T(G_2) \quad V_{G_1} = aV_{G_2} \]

\begin{proof}
    证明:上述变换只改变了赢得矩阵元素的数值,不改变相对大小关系。
\end{proof}

如果博弈问题具有如下对称性:

\[ \mathbf{A} = -\mathbf{A}^T \]

自身角度的赢得矩阵相同

\[ T_{\text{I}}(G) = T_{\text{II}}(G) \quad a_{ij} = \begin{cases} 
-a_{ji} & i \neq j \\
0 & i = j 
\end{cases} \]

\[ V_G = E(\mathbf{x}^*, \mathbf{x}^*) = \sum_{j=1}^n \sum_{i=1}^m a_{ij} x_i^* x_j^* = -V_G = 0 \]

这场游戏没有赢家,可以近似的理解为双方的损失相同。

互补松驰性方程组

$$
\begin{aligned}
x_i^* > 0 & \quad \Rightarrow \quad \sum_{j=1}^n a_{ij} y_j^* = v^* = E(x^*, y^*) \quad i \in \{1, 2, \cdots, m\} \\
y_j^* > 0 & \quad \Rightarrow \quad \sum_{i=1}^m a_{ij} x_i^* = w^* = E(x^*, y^*) \quad j \in \{1, 2, \cdots, n\}
\end{aligned}
$$

$$
\sum_{i=1}^m x_i^* = 1
$$

$$
\sum_{j=1}^n y_j^* = 1
$$

可以理解为对于均衡解,每条纯策略在对方的混合策略下都是相等且最优的。

\begin{example}
    田忌赛马
    $$
\mathbf{A} = \begin{bmatrix}
-3 & -1 & 1 & -1 & -1 & -1 \\
-1 & -3 & -1 & 1 & -1 & -1 \\
-1 & -1 & -3 & -1 & 1 & -1 \\
-1 & -1 & -1 & -3 & -1 & 1 \\
1 & -1 & -1 & -1 & -3 & -1 \\
-1 & 1 & -1 & -1 & -1 & -3
\end{bmatrix}
$$

任何一条纯策略均是对方特定策略下的唯一赢得策略,\textbf{也就是说都有赢的可能},故有

$$
x_i^* > 0 \quad i = 1, 2, \cdots, m
$$
$$
y_j^* > 0 \quad j = 1, 2, \cdots, n
$$
\end{example}

\begin{remark}
    \textbf{利用线性规划求解博弈问题(需要先假设w,v$\ge$0)}
\end{remark}

\begin{equation}
        \begin{aligned}
        & \text{max} & w \\
        & \text{s.t.} & \sum_{i=1}^{m} a_{ij} x_i & \geq w & j = 1, 2, \cdots, n \\
        & & \sum_{i=1}^{m} x_i & = 1 \\
        & & x_i & \geq 0 & i = 1, 2, \cdots, m
        \end{aligned}
        \end{equation}
        
        \begin{equation}
        x_i' = \frac{x_i}{w}
        \end{equation}
        
        \begin{equation}
        \begin{aligned}
        & \text{min} & \sum_{i=1}^{m} x_i' & = \frac{1}{w} \\
        & \text{s.t.} & \sum_{i=1}^{m} a_{ij} x_i' & \geq 1 & j = 1, 2, \cdots, n \\
        & & x_i' & \geq 0 & i = 1, 2, \cdots, m
        \end{aligned}
        \end{equation}
        
        \begin{equation}
        \begin{aligned}
        & \text{min} & v \\
        & \text{s.t.} & \sum_{j=1}^{n} a_{ij} y_j & \leq v & i = 1, 2, \cdots, m \\
        & & \sum_{j=1}^{n} y_j & = 1 \\
        & & y_j & \geq 0 & j = 1, 2, \cdots, n
        \end{aligned}
        \end{equation}
        
        \begin{equation}
        y_j' = \frac{y_j}{v}
        \end{equation}
        
        \begin{equation}
        \begin{aligned}
        & \text{max} & \sum_{j=1}^{n} y_j' & = \frac{1}{v} \\
        & \text{s.t.} & \sum_{j=1}^{n} a_{ij} y_j' & \leq 1 & i = 1, 2, \cdots, m \\
        & & y_j' & \geq 0 & j = 1, 2, \cdots, n
        \end{aligned}
\end{equation}


囚徒困境的纯策略解
$$
A = \begin{bmatrix} -9 & 0 \\ -15 & -1 \end{bmatrix}, \quad
B = \begin{bmatrix} -9 & -15 \\ 0 & -1 \end{bmatrix}
$$
局中人I选择:
$$ j_i^* (B) = \begin{bmatrix} 1 \\ 1 \end{bmatrix} \quad \Rightarrow \quad i_j^* (A) = 1 \quad \Rightarrow \quad (i^*, j^*) = (1, 1) $$

局中人II选择:
$$ i_j^* (A) = \begin{bmatrix} 1 & 1 \end{bmatrix} \quad \Rightarrow \quad j_i^* (B) = 1 \quad \Rightarrow \quad (i^*, j^*) = (1, 1) $$

\begin{definition}
    Nash均衡点

满足以下条件的策略对 \((\alpha_{i*}, \beta_{j*})\)

\[ a_{i*j*} \geq a_{ij*} \quad i = 1, 2, \cdots, m \]

\[ b_{i*j*} \geq b_{i*j} \quad j = 1, 2, \cdots, n \]

没有一个局中人愿意单方面改变策略
\end{definition}

\ifx\allfiles\undefined
\end{document}
\fi