\ifx\allfiles\undefined
\documentclass[12pt, a4paper, oneside, UTF8]{ctexbook}
\def\path{../config}
\input{../config/_config}
\begin{document}
% \input{../config/cover}
\else
\fi

\chapter{网络优化理论}

\begin{definition}
    网络

    G(图)=(V(节点),E(弧))
\end{definition}

\begin{definition}
    次

    $d(v)$:以顶点v为端点的边的个数
\end{definition}

\begin{definition}
    树时一个连通无圈无向的图

    树T = (V, E),$|V|=n$,$|E|=m$,则下列说法等价:

(1) T是一个树

(2) T无圈,且m=n-1

(3) T连通,且m=n-1

(4) T无圈,但每加一个新边,可得唯一的圈

(5) T连通,但每舍去一条边即不连通

(6) T中任意两点,有唯一的链相连
\end{definition}

实际上可以管这个叫最小生成树,因为你找不出更小一个连通无圈无向的图了。对于一个图的最小生成树,我们可以采用破圈法。

\begin{definition}
    \textbf{生成子图}:给定图 \( G = (V, E) \),其生成子图 \( G' = (V', E') \) 满足以下条件:
    \begin{itemize}
        \item \( V' \subseteq V \):子图的顶点集合是原图顶点集合的子集;
        \item \( E' \subseteq E \):子图的边集合是原图边集合的子集;
        \item \( E' \) 仅包含连接 \( V' \) 中顶点的边。
    \end{itemize}
\end{definition}

\begin{remark}
    几个概念的辨析

    \begin{enumerate}
        \item 链:无方向要求,可以反向链接
        \item 道路:有方向要求,不能反向链接
        \item 圈:顾名思义,但是可以反向链接
        \item 回路:有方向要求的圈,不能反向链接
    \end{enumerate}
\end{remark}

\begin{enumerate}
    \item 对于所有权值都非负的图,最短路可用dijkstra算法求解
    \item 对于含负权值的图,最短路可用逐次逼近法求解
    \item 对于任意两点最短距离,可以使用Floyd算法求解
\end{enumerate}

何老师的问题:对于逐次逼近法,最多需要几次迭代。

如果不能提早收敛的话,得要n-1次(n是节点个数)

\textbf{求解最短路的两种思路:}

思路1:两点间用逐次逼近法求解(这也解释了为什么迭代某次发现不变之后就是收敛了)

\[ d_{ij}^{(k)} = \min_{l} \{ d_{il}^{(k-1)} + l_{lj} \} \quad l=1,2,\ldots,n \quad 1 \leq k \leq n \]

思路2:\( l_{ij} \)用上一步迭代结果替代

\[ d_{ij}^{(k)} = \min_{l} \{ d_{il}^{(k-1)} + d_{lj}^{(k-1)} \} \quad l=1,2,\ldots,n \quad 1 \leq k \leq n \]

这就是Floyd算法的思路

\ifx\allfiles\undefined
\end{document}
\fi