\ifx\allfiles\undefined
\documentclass[12pt, a4paper, oneside, UTF8]{ctexbook}
\def\path{../config}
\usepackage{amsmath}
\usepackage{amsthm}
\usepackage{array}
\usepackage{amssymb}
\usepackage{graphicx}
\usepackage{mathrsfs}
\usepackage{enumitem}
\usepackage{geometry}
\usepackage[colorlinks, linkcolor=black]{hyperref}
\usepackage{stackengine}
\usepackage{yhmath}
\usepackage{extarrows}
% \usepackage{unicode-math}
\usepackage{esint}
\usepackage{multirow}
\usepackage{fancyhdr}
\usepackage[dvipsnames, svgnames]{xcolor}
\usepackage{listings}
\usepackage{float} % Required for the H float option
\definecolor{mygreen}{rgb}{0,0.6,0}
\definecolor{mygray}{rgb}{0.5,0.5,0.5}
\definecolor{mymauve}{rgb}{0.58,0,0.82}
\definecolor{NavyBlue}{RGB}{0,0,128}
\definecolor{Rhodamine}{RGB}{255,0,255}
\definecolor{PineGreen}{RGB}{0,128,0}

\graphicspath{ {figures/},{../figures/}, {config/}, {../config/} }

\linespread{1.6}

\geometry{
    top=25.4mm, 
    bottom=25.4mm, 
    left=20mm, 
    right=20mm, 
    headheight=2.17cm, 
    headsep=4mm, 
    footskip=12mm
}

\setenumerate[1]{itemsep=5pt,partopsep=0pt,parsep=\parskip,topsep=5pt}
\setitemize[1]{itemsep=5pt,partopsep=0pt,parsep=\parskip,topsep=5pt}
\setdescription{itemsep=5pt,partopsep=0pt,parsep=\parskip,topsep=5pt}

\lstset{
    language=Mathematica,
    basicstyle=\tt,
    breaklines=true,
    keywordstyle=\bfseries\color{NavyBlue}, 
    emphstyle=\bfseries\color{Rhodamine},
    commentstyle=\itshape\color{black!50!white}, 
    stringstyle=\bfseries\color{PineGreen!90!black},
    columns=flexible,
    numbers=left,
    numberstyle=\footnotesize,
    frame=tb,
    breakatwhitespace=false,
} 

\lstset{
    language=TeX, % 设置语言为 TeX
    basicstyle=\ttfamily, % 使用等宽字体
    breaklines=true, % 自动换行
    keywordstyle=\bfseries\color{NavyBlue}, % 关键字样式
    emphstyle=\bfseries\color{Rhodamine}, % 强调样式
    commentstyle=\itshape\color{black!50!white}, % 注释样式
    stringstyle=\bfseries\color{PineGreen!90!black}, % 字符串样式
    columns=flexible, % 列的灵活性
    numbers=left, % 行号在左侧
    numberstyle=\footnotesize, % 行号字体大小
    frame=tb, % 顶部和底部边框
    breakatwhitespace=false % 不在空白处断行
}

% \begin{lstlisting}[language=TeX] ... \end{lstlisting}

% 定理环境设置
\usepackage[strict]{changepage} 
\usepackage{framed}

\definecolor{greenshade}{rgb}{0.90,1,0.92}
\definecolor{redshade}{rgb}{1.00,0.88,0.88}
\definecolor{brownshade}{rgb}{0.99,0.95,0.9}
\definecolor{lilacshade}{rgb}{0.95,0.93,0.98}
\definecolor{orangeshade}{rgb}{1.00,0.88,0.82}
\definecolor{lightblueshade}{rgb}{0.8,0.92,1}
\definecolor{purple}{rgb}{0.81,0.85,1}

\theoremstyle{definition}
\newtheorem{myDefn}{\indent Definition}[section]
\newtheorem{myLemma}{\indent Lemma}[section]
\newtheorem{myThm}[myLemma]{\indent Theorem}
\newtheorem{myCorollary}[myLemma]{\indent Corollary}
\newtheorem{myCriterion}[myLemma]{\indent Criterion}
\newtheorem*{myRemark}{\indent Remark}
\newtheorem{myProposition}{\indent Proposition}[section]

\newenvironment{formal}[2][]{%
	\def\FrameCommand{%
		\hspace{1pt}%
		{\color{#1}\vrule width 2pt}%
		{\color{#2}\vrule width 4pt}%
		\colorbox{#2}%
	}%
	\MakeFramed{\advance\hsize-\width\FrameRestore}%
	\noindent\hspace{-4.55pt}%
	\begin{adjustwidth}{}{7pt}\vspace{2pt}\vspace{2pt}}{%
		\vspace{2pt}\end{adjustwidth}\endMakeFramed%
}

\newenvironment{definition}{\vspace{-\baselineskip * 2 / 3}%
	\begin{formal}[Green]{greenshade}\vspace{-\baselineskip * 4 / 5}\begin{myDefn}}
	{\end{myDefn}\end{formal}\vspace{-\baselineskip * 2 / 3}}

\newenvironment{theorem}{\vspace{-\baselineskip * 2 / 3}%
	\begin{formal}[LightSkyBlue]{lightblueshade}\vspace{-\baselineskip * 4 / 5}\begin{myThm}}%
	{\end{myThm}\end{formal}\vspace{-\baselineskip * 2 / 3}}

\newenvironment{lemma}{\vspace{-\baselineskip * 2 / 3}%
	\begin{formal}[Plum]{lilacshade}\vspace{-\baselineskip * 4 / 5}\begin{myLemma}}%
	{\end{myLemma}\end{formal}\vspace{-\baselineskip * 2 / 3}}

\newenvironment{corollary}{\vspace{-\baselineskip * 2 / 3}%
	\begin{formal}[BurlyWood]{brownshade}\vspace{-\baselineskip * 4 / 5}\begin{myCorollary}}%
	{\end{myCorollary}\end{formal}\vspace{-\baselineskip * 2 / 3}}

\newenvironment{criterion}{\vspace{-\baselineskip * 2 / 3}%
	\begin{formal}[DarkOrange]{orangeshade}\vspace{-\baselineskip * 4 / 5}\begin{myCriterion}}%
	{\end{myCriterion}\end{formal}\vspace{-\baselineskip * 2 / 3}}
	

\newenvironment{remark}{\vspace{-\baselineskip * 2 / 3}%
	\begin{formal}[LightCoral]{redshade}\vspace{-\baselineskip * 4 / 5}\begin{myRemark}}%
	{\end{myRemark}\end{formal}\vspace{-\baselineskip * 2 / 3}}

\newenvironment{proposition}{\vspace{-\baselineskip * 2 / 3}%
	\begin{formal}[RoyalPurple]{purple}\vspace{-\baselineskip * 4 / 5}\begin{myProposition}}%
	{\end{myProposition}\end{formal}\vspace{-\baselineskip * 2 / 3}}


\newtheorem{example}{\indent \color{SeaGreen}{Example}}[section]
\renewcommand{\proofname}{\indent\textbf{\textcolor{TealBlue}{Proof}}}
\newenvironment{solution}{\begin{proof}[\indent\textbf{\textcolor{TealBlue}{Solution}}]}{\end{proof}}

% 自定义命令的文件

\def\d{\mathrm{d}}
\def\R{\mathbb{R}}
%\newcommand{\bs}[1]{\boldsymbol{#1}}
%\newcommand{\ora}[1]{\overrightarrow{#1}}
\newcommand{\myspace}[1]{\par\vspace{#1\baselineskip}}
\newcommand{\xrowht}[2][0]{\addstackgap[.5\dimexpr#2\relax]{\vphantom{#1}}}
\newenvironment{mycases}[1][1]{\linespread{#1} \selectfont \begin{cases}}{\end{cases}}
\newenvironment{myvmatrix}[1][1]{\linespread{#1} \selectfont \begin{vmatrix}}{\end{vmatrix}}
\newcommand{\tabincell}[2]{\begin{tabular}{@{}#1@{}}#2\end{tabular}}
\newcommand{\pll}{\kern 0.56em/\kern -0.8em /\kern 0.56em}
\newcommand{\dive}[1][F]{\mathrm{div}\;\boldsymbol{#1}}
\newcommand{\rotn}[1][A]{\mathrm{rot}\;\boldsymbol{#1}}

% 修改参数改变封面样式,0 默认原始封面、内置其他1、2、3种封面样式
\def\myIndex{0}


\ifnum\myIndex>0
    \input{\path/cover_package_\myIndex}
\fi

\def\myTitle{标题:一份LaTeX笔记模板}
\def\myAuthor{作者名称}
\def\myDateCover{封面日期: \today}
\def\myDateForeword{前言页显示日期: \today}
\def\myForeword{前言标题}
\def\myForewordText{
    
    这是一个基于\LaTeX{}的模板,用于撰写学习笔记。

    模板旨在提供一个简单、易用的框架,以便你能够专注于内容,而不是排版细节,如不是专业者,不建议使用者在模板细节上花费太多时间,而是直接使用模板进行笔记撰写。遇到问题,再进行调整解决。
}
\def\mySubheading{副标题}


\begin{document}
% \input{\path/cover_text_\myIndex.tex}

\newpage
\thispagestyle{empty}
\begin{center}
    \Huge\textbf{\myForeword}
\end{center}
\myForewordText
\begin{flushright}
    \begin{tabular}{c}
        \myDateForeword
    \end{tabular}
\end{flushright}

\newpage
\pagestyle{plain}
\setcounter{page}{1}
\pagenumbering{Roman}
\tableofcontents

\newpage
\pagenumbering{arabic}
\setcounter{chapter}{-1}
\setcounter{page}{1}

\pagestyle{fancy}
\fancyfoot[C]{\thepage}
\renewcommand{\headrulewidth}{0.4pt}
\renewcommand{\footrulewidth}{0pt}








\else
\fi

% ##########

\chapter{常用定理环境使用}
\section{基础使用}

\textbf{定义}环境的使用,\textcolor{OrangeRed}{定义环境单独编号}
\begin{lstlisting}[language=TeX]
    \begin{defn}[名称、可不写]
        % content
    \end{defn}
\end{lstlisting}

\begin{definition}
    设非空集合$X,\;Y$满足对应法则$f$,对$X$中任意一个元素$x$,按法则$f$,在$Y$中都有唯一确定的元素$y$与之对应,那么称$f$为$X$到$Y$的映射,记$f:X \to Y$.
\end{definition}

\myspace{1}

\textbf{定理}环境的使用,\textcolor{OrangeRed}{定理、引理、准则公用一个编号}
\begin{lstlisting}[language=TeX]
    \begin{thm}[名称、可不写]
        % content
    \end{thm}
\end{lstlisting}

\begin{theorem}[拉格朗日中值定理]
    若函数$f(x)$满足,在闭区间$[a,b]$上连续、在开区间$(a,b)$内可导,那么在$(a,b)$上至少有一点$\xi$($a<\xi<b$),使得
    \begin{equation}\label{eq:3-1}
        f(b)-f(a) = f'(\xi)(b-a)
    \end{equation}
    成立.
\end{theorem}

\myspace{1}

\textbf{引理}环境的使用
\begin{lstlisting}[language=TeX]
    \begin{lemma}[名称、可不写]
        % content
    \end{lemma}
\end{lstlisting}

\begin{lemma}[费马引理]
    函数$f(x)$在点$x_0$的某邻域内有定义,并且在点$x_0$处可导,如果对于任意$x \in U(x_0)$,都有$f(x)\le f(x_0)$或$f(x)\ge f(x_0)$,那么$f'(x_0) = 0$.
\end{lemma}

\myspace{1}

\textbf{推论}环境的使用
\begin{lstlisting}[language=TeX]
    \begin{corollary}[名称、可不写]
        % content
    \end{corollary}
\end{lstlisting}

\begin{corollary}
    如果在区间$[a,b]$上$f(x)\le g(x)$,那么有
    \[
        \int_{a}^{b}f(x)\d x \le \int_{a}^{b}g(x)\d x
    \]
\end{corollary}

\myspace{1}

\textbf{准则}环境的使用
\begin{lstlisting}[language=TeX]
    \begin{criterion}[名称、可不写]
        % content
    \end{criterion}
\end{lstlisting}

\begin{criterion}[夹逼准则]
    若$x \in \mathring{U}(x_0)$(或$|x|>M$),有$g(x)\le f(x) \le h(x)$,且
    \[
        \lim g(x) = \lim h(x) = A
    \]
    那么$\lim f(x)=A$.
\end{criterion}

\myspace{1}

\textcolor{OrangeRed}{说明、解、证明环境不编号,命题、例题独立编号}

\textbf{例题、解、证明}环境的使用
\begin{lstlisting}[language=TeX]
    \begin{rmk}[名称、可不写]
        % content
    \end{rmk}
\end{lstlisting}

\begin{remark}
    这是一段说明
\end{remark}

\myspace{1}

\textbf{命题}环境的使用
\begin{lstlisting}[language=TeX]
    \begin{proposition}[名称、可不写]
        % content
    \end{proposition}
\end{lstlisting}

\begin{proposition}
    这是一段命题
\end{proposition}

\myspace{1}

\textbf{例题、解、证明}环境的使用
\begin{lstlisting}[language=TeX]
    \begin{example}
        % content
    \end{example}

    \begin{solution}
        % content
    \end{solution}
\end{lstlisting}

\begin{example}
    设积分$\displaystyle\int_C xy^2\d x +y \varphi(x)\d y$与路径无关,其中$\varphi$有连续导数,$C$是点$(0,0)$到点$(1,1)$的线段,且$\varphi(0)=0$,计算这个积分.
\end{example}
\begin{solution}
    记$P = xy^2,\; Q = y\varphi(x)$,则
    \[
        P_y =2xy ,\enspace Q_x = y\varphi'(x)
    \]
    而$Q_x = P_y$,则有$\varphi'(x) = 2x$,两边积分有$\varphi(x) = x^2+C$,又$\varphi(0)=0$,于是$C=0$,即$\varphi(x) = x^2$,所求积分
    \[
        \int_{(0,0)}^{(1,1)} xy^2\d x +y x^2\d y  = \int_0^1 y \d y = \frac{1}{2}
    \]
\end{solution}

\begin{proof}
    记$P = xy^2,\; Q = y\varphi(x)$,则...
\end{proof}

% #########

\ifx\allfiles\undefined
\end{document}
\fi